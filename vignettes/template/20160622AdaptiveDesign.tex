\documentclass{beamer}
% applying a beamer theme
\usetheme{Antibes}
% adding an informative footline
\useoutertheme{infolines}

\usepackage{natbib}
\usepackage{hyperref}
\usepackage{verbatim}
\usepackage{natbib}
\usepackage{graphicx}
\usepackage{appendixnumberbeamer}

\usepackage{pgfpages}
%\pgfpagesuselayout{2 on 1}[letterpaper,border shrink=5mm]

\setbeameroption{hide notes}
%\setbeameroption{show notes}
%\setbeameroption{show only notes}


% items enclosed in square brackets are optional; explanation below
\title[BDM Seminar]{Interim Analysis and Adaptive Design}
%\subtitle[Errors]{Estimation of numerical errors}
\logo{\includegraphics[width=2cm,height=2cm,keepaspectratio]{logo.png}}
\author[R. Qin and Y. Du]{Rui Qin and Yunling Du}
\institute[]{Biostatistics and Data Management \\
Regeneron Pharmaceuticals \\ 
\texttt{rui.qin@regeneron.com}
}

\date[June 22, 2016]{June 22, 2016}


\begin{document}

%--------------------- the title page frame -------------------------%
\begin{frame}[plain]
  \titlepage
\end{frame}
%--------------------- the title page frame -------------------------%
%\begin{comment}
%\end{comment}

%--------------------- create table of contents page ----------------%
\begin{frame}{Table of Contents}
  \tableofcontents
\end{frame}

\section{Adaptive Design}
%----------- slide --------------------------------------------------%
\begin{frame}
	\frametitle{Clinical Trial}
	\begin{itemize}
		\item A prospectively planned experiment for the purpose of evaluating a potentially beneficial therapy or treatment
		\item Conducted under as many controlled conditions as possible so that they provide definitive answers to pre-determined, well-defined questions
		\item Classic design requires such parameters to be pre-specified and fixed throughout a clinical trial
		\begin{itemize}
		  \item Sample size
		  \item Randomization ratio
		  \item Number of study arms
		  \item \ldots\ \ldots
		\end{itemize}
	\end{itemize}
\end{frame}
\note{generally no interim analysis, traditional into four phases: phase I, phase II, phase III and phase IV.}
%----------- slide --------------------------------------------------%
\begin{frame}
    \frametitle{Adaptations in Clinical Trial}
Sometimes are necessary to     
    \begin{itemize}
        \item reflect real medical practice on the actual patient population with the disease under study
        \item increase the probability of success for identifying clinical benefit of treatment  
    \end{itemize}
Include but not limited to         
  \begin{itemize}
  \item Modifications of inclusion/exclusion criteria
  \item Adjustment of study dose or treatment
  \item Extension of study duration
  \item Changes in study endpoints
  \item \textcolor{red}{Modifications in study design based on interim analysis}
  \end{itemize}

\end{frame}
\note{some adaptations are more statistical}

%----------- slide --------------------------------------------------%
\begin{frame}
	\frametitle{Interim Analysis (IA)}
	\begin{itemize}
		\item \emph{``Any examination of data obtained in a study while that study is still ongoing, and is not restricted to cases in which there are formal between-group comparisons"} -- FDA Guidance on Adaptive Design (2010)
		\item Reasons for interim analysis \begin{itemize}
			\item Ethical
			\note{ensure patients not exposed to unsafe, inferior or ineffective treatment regimens}
			\item Administrative
			\note{ensure the trial is being executed as planned, correct population, eligibility, treatment prescribed correctly, check on assumptions made about the study}
			\item Economic
			\note{developed in order to obtain economic benefit, positive results to move to next phase, negative to ensure resources are not wasted}
		\end{itemize}
		\item Types of interim analysis \begin{itemize}
			\item Efficacy vs. Safety vs. Other
			\item Blinded vs. Unblinded
		\end{itemize}
		\item Multiple stages are formed with interim analysis
		\note{While continuous monitoring is desirable but often impractical}
   		\end{itemize}
\end{frame}

%----------- slide --------------------------------------------------%
\begin{frame}
	\frametitle{Adaptive Design}
  \begin{itemize}
	\item \emph{``A study design that includes a prospectively planned opportunity for modification of one or more specified aspects of the study design and hypotheses based on analysis of data"} -- FDA Guidance on Adaptive Design (2010)
	\item Adaptations based on interim analysis
	\begin{itemize}
	\item Dose escalation/de-escalation
	\item Early stopping for superiority or futility
	\item Sample size re-estimation
	\item Outcome-adaptive randomization
	\item Study population enrichment
	\item Drop or add study arms
    \item \ldots\ \ldots
	\end{itemize}		
	\end{itemize}
\end{frame}

%----------- slide --------------------------------------------------%
\begin{frame}
	\frametitle{Types of Adaptive Designs}	
	\begin{itemize}
		\item \textcolor{brown}{Adaptive Dose-Finding Design}
		\item \textcolor{purple}{Group Sequential Design} 
		\item \textcolor{blue}{Sample Size Re-estimation}
		\item \textcolor{blue}{Adaptive Randomization Design}
		\item \textcolor{blue}{Drop-Loser and/or Add-Arm Design}
		\item \textcolor{blue}{Biomarker-Adaptive Design}
	    \item \ldots\ \ldots
	    \item \textcolor{red}{Bayesian Design}
	\end{itemize}	
\end{frame}
%----------- slide --------------------------------------------------%
\begin{frame}
	\frametitle{Statistical Aspects}
	\begin{itemize}
		\item Type I error --  $\alpha$ control and determination of stopping boundaries
		\item Type II error -- $\beta$ control and calculation of power or sample size 
		\item Trial monitoring -- make decisions based on conditional power (or futility index)
		\item Analysis after completion of study -- calculation of adjusted p-values, unbiased point estimates and confidence intervals
	\end{itemize}
\end{frame}
%----------- slide --------------------------------------------------%
\begin{frame}
	\frametitle{Pro's and Con's}
	\begin{figure}
		\begin{center}
			\includegraphics[width=0.95\textwidth,height=0.95\textheight,keepaspectratio]{adaptive-pros-and-cons-2.jpg}
			\footnote{www.advancedclinical.com}
		\end{center}
	\end{figure}	
\end{frame}
%----------- slide --------------------------------------------------%
\begin{frame}<presentation:0>
    \frametitle{Benefits of Adaptive Designs}  
\note{Make a clinical study more efficient}  
   \begin{itemize}
      \item Early decision making
      \item Shorter study duration
      \item Fewer patients exposed to a treatment which is inefficacious
      \item More likely to demonstrate an effect of the treatment if one exists
      \item More flexible and informative           
    \end{itemize}
\end{frame}
\addtocounter{framenumber}{-1}

%----------- slide --------------------------------------------------%
\begin{frame}<presentation:0>
	\frametitle{Concerns of Adaptive Design}
  \begin{itemize}
	\item Potential to increase the chance of erroneous positive conclusion
	\item Difficulty in interpreting results when a treatment effect is shown
	\item Operational bias from unblinding interim analysis
	\item Potential to limit identifying gaps in knowledge
	\item Elimination of time to thoughtfully explore study results
	\item Complex adaptive design for increased planning and more advanced time frame for planning
  \end{itemize}
\end{frame}
\addtocounter{framenumber}{-1}
%----------- slide --------------------------------------------------%


\section{Examples}
%----------- slide --------------------------------------------------%
\subsection{Group Sequential Design (GSD)}
\begin{frame}
	\frametitle{Group Sequential Design (GSD)}
	\begin{figure}
		\begin{center}
			\includegraphics[width=0.9\textwidth,height=0.9\textheight,keepaspectratio]{group-sequential-design-v11_EN.png}
			\footnote{Figure 1. of EUPATI (2015)}
		\end{center}
	\end{figure}	
\end{frame}
%----------- slide --------------------------------------------------%
\begin{frame}
	\frametitle{A Phase III NSCLC Trial}
Consider to design a phase III clinical trial for an experimental therapy vs. standard chemotherapy (control) in non-small cell lung cancer (NSCLC) patients, the primary endpoint is overall survival (OS)
  \begin{itemize}
  \item $OS_{ctrl}=12\ months$
  \item The clinically meaningful effect size $HR=0.75$, (i.e. $OS_{trt}=16\ months$) 
  \item Type I error $\alpha = 2.5\%$ (one-sided)
  \item Power $1-\beta = 90\%$
  \item Accrual period of 48 months 
%  \item Drop-out rate \ldots\ \ldots
  \item Minimum follow-up period of 12 months
\end{itemize}

\end{frame}
%----------- slide --------------------------------------------------%
\begin{frame}
	\frametitle{Classic Design with Fixed Sample Size}
\begin{itemize}
		\item Pre-specify accrual and drop out rates
	    \item Total study duration is at least 60 months!
		\item Sample size \begin{itemize}
			\item Required number of events is 507
			\item Required number of patients is 718
		\end{itemize}
        \item No (formal) interim analysis
        \item Must wait till the study end to analyze data and make decisions
\end{itemize}
\end{frame}
%----------- slide --------------------------------------------------%
\begin{frame}
	\frametitle{GSD with Interim Analysis}
Can we evaluate efficacy results earlier to make decisions? \begin{itemize}
  \item If the experimental therapy truly works, can we complete study early to claim efficacy? -- \textcolor{green}{Superiority}
  \item If the experimental therapy does not work, can we terminate study early to avoid harmful patient exposure? -- \textcolor{red}{Futility}
  %\item Both of the above together?
\end{itemize}

\textcolor{blue}{Solution: Group sequential design with interim analysis} 
\end{frame}
%----------- slide --------------------------------------------------%
\begin{frame}
	\frametitle{GSD with Interim Analysis}
 \begin{itemize}
		\item How many interim analysis? \begin{itemize}
		\item \textcolor{blue}{Not too many as interim analysis takes time and efforts} 
		\end{itemize}
		\item When to conduct interim analysis? \begin{itemize}
		\item \textcolor{blue}{Not too early as information may be too limited for making decisions, at least 25\%--35\%}
		\item \textcolor{blue}{Not too late (relative to study duration) as benefit of interim analysis diminishes} 
		\end{itemize}
		\item Types of interim analysis? \begin{itemize}
		\item \textcolor{blue}{Superiority only}
		\item \textcolor{blue}{Futility only}
		\item \textcolor{blue}{Both superiority and futility (binding vs. non-binding)}
		\end{itemize}
	\end{itemize}
	

\end{frame}
%----------- slide --------------------------------------------------%
\begin{frame}
	\frametitle{Statistical Issues}
\begin{itemize}
   \item Repeated significance testing with interim analysis
   \begin{itemize}
   	\item Claim efficacy after 1st interim analysis
   	\item Claim efficacy after 2nd interim analysis if study continues after 1st interim analysis
   	\item \ldots \ldots
   	\item Claim efficacy after final analysis if study continues after all interim analysis
   \end{itemize}
  	\item Multiple looks of superiority inflate family-wise error rate (FWER) of type I error (introduce bias)   \note{The probability of making one or more false discoveries (type I errors) among all the hypotheses when performing multiple hypotheses tests}
  	\item  Multiple looks of futility inflate FWER of type II error (decrease power)
  	\item Implementation of interim analysis for confirmatory trials must be done by an independent data monitoring committee (IMDC)
\end{itemize}
\end{frame}
%----------- slide --------------------------------------------------%
\begin{frame}<presentation:0>
	\frametitle{Statistical Design with Interim Analysis}
How to deal with repeated significance testing \begin{itemize}
			\item How many interim analysis?
			\item Timing of interim analysis? 
		    \item Repeated significance testing
			\item Actions to take after interim analysis? \begin{itemize}
				\item Stop for efficacy (superiority) only
				\item Stop for inefficacy (futility) only: binding vs. non-binding
				\item Stop for either superiority or futility
				\item Continue to the next stage \ldots \ldots
		     \end{itemize} 
	\end{itemize}
\end{frame}\addtocounter{framenumber}{-1}
%----------- slide --------------------------------------------------%
\begin{frame}<presentation:0>
	\frametitle{Family-Wise Error Rate}
	\begin{itemize}
		\item The probability of making one or more false discoveries (type I errors) among all the hypotheses when performing multiple hypotheses tests
	    \item What if no adjustment for statistical design? \begin{itemize}
		   \item Inflation of FWER
		   \item Inflation overall type II error $\longrightarrow$ lower power	
	    \end{itemize}		
		\item FWER control methods
		\begin{itemize}
			\item Pocock Boundary 
			\item O'Brien-Fleming Boundary
			\item Spending function
		\end{itemize}
	\end{itemize}	
\note{Bonferroni approach, overly conservative, non-parametric approach, Nominal p-value}	
\note{If interim analysis is integrated to stop the trial early for Non-binding lower bound. Lower bounds are sometimes considered as guidelines, which may be ignored during the course of the trial. Since Type I error is in	ated if this is the	case, regulators often demand that the lower bounds be ignored when computing Type I error.}
\end{frame}\addtocounter{framenumber}{-1}
%----------- slide --------------------------------------------------%
\begin{frame}<presentation:0>
	\frametitle{Pocock vs. O'Brien-Fleming Boundaries}
	\begin{figure}
		\begin{center}
			\includegraphics[width=1.0\textwidth,height=0.75\textheight,keepaspectratio]{seqd04e.png}
		\end{center}
	\end{figure}
\end{frame}\addtocounter{framenumber}{-1}
\note{By setting the interim boundaries to be substantially higher than(:975) = 1:96 we have
	drastically reduced the excess Type I error caused by multiple testing while still testing at the
	nominal .05 (2-sided) level at the analysis.	
	Constant critical value, constant nominal significance level (well below the overall significance level to avoid multiple-look problem)
	
	Critical value computed numerically using the joint distribution of sequence of test statistics
	Show results of design parameters and boundary plots of p-value and hazard ratio}

\note{More stringent boundary, nominal significance levels needed to reject null hypothesis in each analysis increase as study progress.
	
	More difficult to reject H0 at earlier analyses but easier later on.
	
	In oncology trials, OF boundary is preferred \begin{itemize}
		\item data quality at study entry
		\item data entry timeliness
		\item Scientific community skeptical when there are only a few observations
	\end{itemize}}
%----------- slide --------------------------------------------------%
\begin{frame}
	\frametitle{GSD for NSCLC Trial}
Consider to modify the classic design for NSCLC trial to a group sequential design with \begin{itemize}
		\item One interim analysis at 50\% information (i.e. number of events)
		\item Both superiority and futility at interim analysis
		\item FWER control methods \begin{itemize}
		\item Pocock bounds
		\item O'Brien-Fleming bounds
		\item Spending function approach (Hwang-Shih-DeCani family)
	\end{itemize}
\end{itemize}	
\end{frame}		
%----------- slide --------------------------------------------------%
\begin{frame}
	\frametitle{GSD with Pocock Bounds}
\begin{columns}
	\begin{column}{0.48\textwidth}
	\includegraphics[scale=0.32]{figpocockhr} 
	\end{column}
	\begin{column}{0.48\textwidth} \begin{itemize}
	  \item Terminate study for superiority if $HR \leq 0.79$ at interim analysis
	  \item Terminate study for futility if $HR\geq 0.89 $ at interim analysis
	  \item Continue study if $ 0.79 < HR < 0.89 $ at interim analysis
	  \item Claim efficacy if $HR \leq 0.84$ after final analysis
	  \item Required number of events increases (from 507) to 637 
	  \end{itemize}
	\end{column}
\end{columns}	
\end{frame}
%----------- slide --------------------------------------------------%
\begin{frame}
	\frametitle{GSD with Pocock Bounds}
\begin{table}[ht]
	\centering\small
	\begin{tabular}{llrr}
		\hline
		Analysis & Value & Efficacy & Futility \\ 
		\hline
		IA 1: 50\% & Z & 2.1570 & 1.0313 \\ 
		N: 768 & p (1-sided) & 0.0155 & 0.1512 \\ 
		Events: 319 & HR at bound & 0.7852 & 0.8908 \\ 
		Month: 36 & P(Cross) if $H_{0}$ true (HR=1) & 0.0155 & 0.8488 \\ 
		& P(Cross) if $H_{1}$ true (HR=0.75) & 0.6600 & 0.0620 \\ 
		\hline
		Final & Z & 2.2010 & 2.2010 \\ 
		N: 902 & p (1-sided) & 0.0139 & 0.0139 \\ 
		Events: 637 & HR at bound & 0.8399 & 0.8399 \\ 
		Month: 60 & P(Cross) if $H_{0}$ true (HR=1) & 0.0229 & 0.9771 \\ 
		& P(Cross) if $H_{1}$ true (HR=0.75) & 0.9000 & 0.1000 \\ 
		\hline
	\end{tabular}
	\label{tab1}
\end{table}
\begin{itemize}
	\item Approx. 66\% chance to claim superiority at IA if therapy is efficacious
	\item Approx. 85\% chance to claim futility at IA if therapy is not efficacious 
    \item Duration of study reduced to 36 months if either superiority or futility is claimed at IA
\end{itemize}
\end{frame}


%----------- slide --------------------------------------------------%
\begin{frame}
	\frametitle{GSD with O'Brien-Fleming Bounds}
\begin{columns}
	\begin{column}{0.48\textwidth}
	\includegraphics[scale=0.32]{figofhr} 
	\end{column}
	\begin{column}{0.48\textwidth} \begin{itemize}
			\item Terminate study for superiority if $HR \leq 0.69$ at interim analysis
			\item Terminate study for futility if $HR \geq 0.97$ at interim analysis
			\item Continue study if $ 0.69 < HR < 0.97 $ at interim analysis
			\item Claim efficacy if $HR \leq 0.84$ after final analysis
			\item Required number of events increases (from 507) to 520
		\end{itemize}
	\end{column}
\end{columns}
\end{frame}

%----------- slide --------------------------------------------------%
\begin{frame}
	\frametitle{GSD with O'Brien-Fleming Bounds}
\begin{table}[ht]
	\centering\small
	\begin{tabular}{llrr}
		\hline
		Analysis & Value & Efficacy & Futility \\ 
		\hline
		IA 1: 50\% & Z & 2.9626 & 0.2670 \\ 
		N: 626 & p (1-sided) & 0.0015 & 0.3947 \\ 
		Events: 260 & HR at bound & 0.6923 & 0.9674 \\ 
		Month: 36 & P(Cross) if $H_{0}$ true (HR=1) & 0.0015 & 0.6053 \\ 
		& P(Cross) if $H_{1}$ true (HR=0.75) & 0.2604 & 0.0200 \\ 
		\hline
		Final & Z & 1.9686 & 1.9686 \\ 
		N: 736 & p (1-sided) & 0.0245 & 0.0245 \\ 
		Events: 520 & HR at bound & 0.8413 & 0.8413 \\ 
		Month: 60 & P(Cross) if $H_{0}$ true (HR=1) & 0.0243 & 0.9757 \\ 
		& P(Cross) if $H_{1}$ true (HR=0.75) & 0.9000 & 0.1000 \\ 
		\hline
	\end{tabular}
	\label{tab4}
\end{table}
\begin{itemize}
	\item Approx. 26\% chance to claim superiority at IA if therapy is efficacious 
	\item Approx. 61\% chance to claim futility at IA if therapy is not efficacious 
	\item Duration of study reduced to 36 months if either superiority or futility is claimed at IA
\end{itemize}
	\end{frame}	
	
%----------- slide --------------------------------------------------%
\begin{frame}
	\frametitle{Pocock vs O'Brien-Fleming Bounds}
\begin{table}[ht]
	\renewcommand{\arraystretch}{1.25}
	\centering
	\begin{tabular}{ccc}
 Design Parameters & Pocock & O'Brien-Fleming\\ 
 \hline
 HR bounds at IA & (0.79,\ 0.89) & (0.69,\ 0.97) \\
 $\alpha$ spending at IA & 0.0155 & 0.0015 \\
 Pr(stop for superiority) at IA & 66\% & 26\% \\
 Pr(stop for futility) at IA & 85\% & 61\% \\
 Events/Sample Size & 637 / 902 & 520 / 736 \\ 
 \hline		
\end{tabular}
\end{table}	
\begin{itemize}
	\item Pocock bounds spends more $\alpha$ at IA , thus more aggressive to claim superiority/futility 
	\item O'Brien-Fleming bounds is more conservative in claiming efficacy/futility at IA, reserving more $\alpha$ for final analysis
	\item O'Brien-Fleming bounds requires fewer event/sample size than that of Pocock bounds 
\end{itemize}
\end{frame}

%----------- slide --------------------------------------------------%
\begin{frame}
	\frametitle{Flexible GSD with Spending Function}
\begin{columns}
	\begin{column}{0.48\textwidth}
		\includegraphics[scale=0.32]{hsdsf} 
	\end{column}
	\begin{column}{0.48\textwidth}	
	\begin{itemize}
		%\item Pocock and O'Brien-Fleming bounds require (approximately) equal stage size
		\item Balance of aggressive/conservative IA
		\item Spending $\alpha$ as a function of the observed information levels
		\item Interim analysis may occur at any times with spending function
		\item Number of interim analyses may change
		\item Operational and logistical restrictions
	\end{itemize}
	\end{column}
\end{columns}
	\note{Administratively it is convenient to schedule interim analyses at fixed calendar times, but if patients are recruited at an uneven rate, the number of new observations between analyses will vary.}
\end{frame}
%----------- slide --------------------------------------------------%
\begin{frame}<presentation:0>
	\frametitle{GSD with Spending Function}
\begin{itemize}
	\item Pocock and O'Brien-Fleming bounds require (approximately) equal group size
	\item Preferred interim analysis may occur at different times
	\item Number of interim analyses may change
	\item Spending the Type I error as a function of the observed information levels
	\begin{itemize}
		\item Kim-DeMets (power) spending function
		\item Hwang-Shih-DeCani spending function
		\item Exponential spending function
		\item \ldots \ldots
	\end{itemize}
	\item Similarly, spending Type II error as a function of the observed information levels 	
\end{itemize}
\note{Administratively it is convenient to schedule interim analyses at fixed calendar times, but if patients are recruited at an uneven rate, the number of new observations between analyses will vary.}
\end{frame}\addtocounter{framenumber}{-1}

%----------- slide --------------------------------------------------%
\begin{frame}<presentation:0>
	\frametitle{Conditional Power}
\begin{itemize}
	\item The probability of rejecting null hypothesis during the rest of the trial based on accumulated data at interim analysis
	\item Futility index is the conditional probability of accepting the null hypothesis
	\item Commonly used for monitoring an ongoing trial, maybe used for sample size re-estimation
	\item A useful concept for communicating with clinical investigator
\end{itemize}
\end{frame}\addtocounter{framenumber}{-1}
%----------- slide --------------------------------------------------%
\begin{frame}
	\frametitle{Conditional Power}
Given a normal test statistic from IA, the conditional power curves under observed effect size (ES), $H_{0}$ and  $H_{1}$
\begin{columns}
	\begin{column}{0.48\textwidth}
		\includegraphics[scale=0.32]{condpower.pdf} 
	\end{column}
	\begin{column}{0.48\textwidth}	
		\begin{itemize}
			\item Probability of rejecting $H_{0}$ (claim efficacy) during the rest of the trial based on accumulated data at IA
			\item Commonly used for monitoring an ongoing trial
			\item Maybe utilized for sample size re-estimation
		\end{itemize}
	\end{column}
\end{columns}	
\end{frame}
%----------- slide --------------------------------------------------%
\subsection{Sample Size Re-estimation (SSR)}
%----------- slide --------------------------------------------------%
\begin{frame}
	\frametitle{Sample Size Re-estimation}
	\begin{figure}
		\begin{center}
			\includegraphics[width=0.95\textwidth,height=0.70\textheight,keepaspectratio]{promising.jpg}
			\footnote{Cytel Presentation (2012)}
		\end{center}
	\end{figure}	
\end{frame}
%----------- slide --------------------------------------------------%
\begin{frame}
	\frametitle{Sample Size Calculation}
\begin{itemize}
	\item Sample size calculation based on early phase trial results or historical data at the design stage \begin{itemize}
	  \item A clinically meaningful effect size
	  \item Variability associated with the effect size (and other nuisance parameters)
	\end{itemize}
	\item What if the effect size and/or the associated variability were incorrectly specified in the NSCLC trial? \begin{itemize}
	\item If $OS_{ctrl}=15$ months and $OS_{trt}=20$ months (still HR=0.75)
	\item \textcolor{blue}{number of event 507, sample size 795 to achieve 90\% power}
	\item If $OS_{ctrl}=12$ months and $OS_{trt}=15$ months (HR=0.80)
	\item \textcolor{blue}{number o event 844, sample size 1300 to achieve 90\% power}
   \end{itemize}
\end{itemize}
\end{frame}
%----------- slide --------------------------------------------------%
\begin{frame}
	\frametitle{Sample Size Re-estimation}
 Can we plan a sample size re-estimation after interim analysis to overcome under-power or over-power in initial design of the NSCLC trial? \begin{itemize}
 	\item If the study is under-powered based on interim analysis, increase sample size
 	\item If the study is over-powered based on interim analysis, reduce sample size (though rarely done)
 \end{itemize}
 
 \textcolor{blue}{Solution: Sample size re-estimation after interim analysis}
 \begin{itemize}
	\item N-adjusted clinical trial design (straightforward)
	\item Integrated with group sequential design (complex)
    \item Types of sample size re-estimation based on interim analysis \begin{itemize}
       \item Blinded
       \item Unblinded
\end{itemize}
\end{itemize}
\end{frame}

%----------- slide --------------------------------------------------%
\begin{frame}
	\frametitle{Blinded SSR for NSCLC Trial}
Consider to modify the classic design for NSCLC trial to a sample size re-estimation design with 
\begin{itemize}
  \item Sample size re-estimation after an interim analysis at 50\% information (i.e. number of events)
  \item Interim analysis is blinded without any knowledge of treatment assignment
  \item Interim analysis is not intended for superiority or futility
  \item Significance level does not need to be adjusted for blinded interim analysis
\end{itemize}
\end{frame}
%----------- slide --------------------------------------------------%
\begin{frame}
	\frametitle{Blinded SSR for NSCLC Trial}
	\begin{itemize}
		\item Specified maximum sample size inflation was 100\%
		\item Assumed enrollment overrun at interim analysis was 25 patients
		\item Observed unblinded median $OS = 17.5$ months at the interim analysis 
		\item A heuristic calculation \vspace{8pt}
		\begin{centering}
		\begin{tabular}{cccc}
			Stage & IA & FA & SSR \\ \hline
			No. of Events & 254 & 507 & 507 \\
			Sample Size & 359 & 718 & 794 \\
			Overrun & 25 & 0 & 0 \\ \hline
		\end{tabular}
		\end{centering}
		\item Hence, blinded SSR suggest to increase sample size (initial design) by $794-718=76$ patients for final analysis (FA)
	\end{itemize}
\end{frame}
%----------- slide --------------------------------------------------%
\begin{frame}
	\frametitle{Unblinded SSR for NSCLC Trial}
	\begin{itemize}
		\item May provide more accurate sample-size estimation based on the estimated effect size at interim analysis
		\item Bias results from knowledge of observed effect size at interim analysis
		\item Statistical approaches to control FWER \begin{itemize}
			    \item Combination test
				\item Conditional error function
				\item Conditional power (CP)
			\end{itemize}
	\end{itemize}
\end{frame}
%----------- slide --------------------------------------------------%
\begin{frame}
	\frametitle{GSD with Unblinded SSR}
\begin{table}[ht]
	\caption{Asymmetric two-sided group sequential design with non-binding futility bound, sample size 833. Efficacy bounds derived using a HSD spending function with gamma = -4. Futility bounds derived using a HSD spending function with gamma = 1.} 
	\label{tab5}
		\centering\small
	\begin{tabular}{llrr}
		\hline
		Analysis & Value & Efficacy & Futility \\ 
		\hline
		IA 1: 50\% & Z & 2.7500 & 0.9316 \\ 
		N: 417 & p (1-sided) & 0.0030 & 0.1758 \\ 
		& HR at bound & 0.7257 & 0.8971 \\ 
		& P(Cross) if HR=1 & 0.0030 & 0.8242 \\ 
		& P(Cross) if HR=0.75 & 0.3889 & 0.0622 \\ 
		\hline
		Final & Z & 1.9811 & 1.9811 \\ 
		N: 833 & p (1-sided) & 0.0238 & 0.0238 \\ 
		& HR at bound & 0.8493 & 0.8493 \\ 
		& P(Cross) if HR=1 & 0.0211 & 0.9789 \\ 
		& P(Cross) if HR=0.75 & 0.9000 & 0.1000 \\ 
		\hline
	\end{tabular}
\end{table}
\end{frame}
%----------- slide --------------------------------------------------%
\begin{frame}
	\frametitle{GSD with Unblinded SSR}
\begin{itemize}
	\item Maximum sample size inflation is specified as 100\%
	\item Assumed enrollment overrrun at IA is 25 patients
	%\item Fixed design sample size is 718
	\item Promising zone in CP interval $(0.36, 0.9)$ where SSR to be conducted
\end{itemize}	
\end{frame}
%----------- slide --------------------------------------------------%
\begin{frame}
	\frametitle{GSD with Unblinded SSR}
	%	\begin{figure}
	\begin{center}
		\includegraphics[width=0.80\textwidth,height=0.80\textheight,keepaspectratio]{ssr.pdf}
	\end{center}
	%	\end{figure}	
\end{frame}
%----------- slide --------------------------------------------------%
\begin{frame}
	\frametitle{GSD with Unblinded SSR}
Based on the GSD with interim analysis for superiority and futility \begin{itemize}
	\item Futility -- Stop after interim analysis with actual sample size of 417 
	\item Superiority -- Stop after interim analysis with actual sample size of 417
\end{itemize}

Otherwise, based on the conditional power at the interim analysis,
\begin{itemize}
	\item CP $< 0.36$ \textcolor{red}{unfavorable} -- Continue the study after interim analysis without SSR, sample size is still 833
	\item CP $\in [0.36, 0.9]$ \textcolor{blue}{promising zone}  -- Increase sample size $833 < N^{*} \leq 833\times 2=1666$ 
	\item CP $> 0.9$ \textcolor{green}{favorable} -- Continue the study after interim analysis without SSR, sample size is still 833
\end{itemize}	
\end{frame}
%----------- slide --------------------------------------------------%
\begin{frame}
	\frametitle{Blinded vs. Unblinded SSR}
\begin{table}[ht]
	\renewcommand{\arraystretch}{1.25}
	\centering
	\begin{tabular}{ccc}
		Design Parameters & Blinded & Unblinded		\\ \hline
		FWER Control & No adjustment & Adjustment \\
		Stat Methods & Straightforward & Complex \\		
		Implementation & In-house & External \\ 
		FDA guidance & Well-understood & Less well-understood \\\hline
	\end{tabular}
\end{table}	
\end{frame}


%----------- slide --------------------------------------------------%
\begin{frame}<presentation:0>
	\frametitle{SSR Summary}
Pros
	\begin{itemize}
		\item Can decide to alter trial size based on partial data or new, external information
		\item Address uncertainty at the design stage
		\item Offers flexibility and reduce upfront resource commitment
		\end{itemize}
Cons \begin{itemize}		
		\item Methods used to adapt must be carefully chosen, regulatory scrutiny over methods and partial unblinding, 
		\item May not improve efficiency over group sequential design
		\item Renegotiate budget and request additional drug supply when an increase in sample size is necessary
	\end{itemize}
\end{frame}\addtocounter{framenumber}{-1}
%----------- slide --------------------------------------------------%

\section{Practical Considerations}
%----------- slide --------------------------------------------------%
\begin{frame}
	\frametitle{Regulatory Guidelines}
	\begin{itemize}  
		\item PhRMA (2006) -- Adaptive designs in clinical drug development - an executive summary of the PhRMA working group
		\item EMA (2007) -- Reflection paper on methodological issues in confirmatory clinical trials planned with an adaptive design
		\item FDA (2010) -- Guidance for the use of Bayesian statistics in medical device clinical trials
		\item \textcolor{purple}{FDA (2010) -- Adaptive design clinical trials for drugs and biologics} 
		\item FDA (2015) -- Adaptive designs for medical device clinical studies
	\end{itemize}  
	
\end{frame}

%----------- slide --------------------------------------------------%
\begin{frame}
	\frametitle{FDA \textcolor{red}{Draft} Guideline 2010}
	\begin{itemize}  
		\item Distributed in February, 2010, expect to publish final document in 2017
		\item Endorsed by both CDER and CBER for drugs and biologics
		\item Well-understood designs \begin{itemize}
			\item Group sequential design
			\item Sample size re-estimation with blinded interim analysis
		\end{itemize}
		\item Less well-understood designs \begin{itemize}
			\item Adaptive dose-selection, sample size re-estimation with unblinded interim analysis, adaptive randomization, adaptive population, endpoint selection, \ldots\ \ldots
		\end{itemize}
	\end{itemize} 	
\end{frame}

%----------- slide --------------------------------------------------%
\begin{frame}
	\frametitle{Challenges}
	\begin{itemize}
		\item Requirements of pre-specified vs. unplanned adaptations
		\item Timing of interim analysis vs. patient accrual
		\item Time and efforts in designing a complex adaptive design
	\end{itemize}
\end{frame}

%----------- slide --------------------------------------------------%
\begin{frame}
	\frametitle{Operations}
	\begin{itemize}
		\item Early interaction with FDA
		\item Extensive simulation studies for evaluation
		\item Documentation in protocol and SAP
		\item Available software and/or packages
	\end{itemize}
\end{frame}
%----------- slide --------------------------------------------------%
\begin{frame}<presentation:0>
	\frametitle{Resources}
	\begin{itemize}
		\item Statistical literatures
		\item PROC SEQDESIGN and PROC SEQTEST by SAS
       \item nTrim by Statistical Solutions 
		\item EAST \texttt{Adapt} add-on package by Cytel
		\item Adaptive design packages (such as gsDesign) in R
		\item ADDPLAN by ICON
	\end{itemize}
\end{frame}\addtocounter{framenumber}{-1}
%----------- slide --------------------------------------------------%
\begin{frame}<presentation:0>
	\frametitle{Operation}
	\begin{itemize}
		\item Internal: Eli Lilly, Novartis method/simulation group
		\item External: Berry Consultants, Cytel
	\end{itemize}
\end{frame}\addtocounter{framenumber}{-1}
%----------- slide --------------------------------------------------%
\begin{frame}
  \frametitle{Summary}
\begin{itemize}
  \item Not intended for adaptations due to poor planning in design stage
  \item Improves efficiency when used appropriately
  \item Currently more acceptable in early-phase drug development when information is limited
  \item Important to communicate with clinical colleagues and FDA
\end{itemize}
\end{frame}
%----------- slide --------------------------------------------------%
\begin{frame}
	\frametitle{References}
\begin{itemize}
\item Jennison and Turnbull (2000) \emph{Group Sequential Methods with Applications to Clinical Trials}, Chapman \& Hall 

\item Chang (2008) \emph{Adaptive Design Theory and Implementation using SAS and R},  Chapman \& Hall 

\item Mehta and Pocock (2009) \emph{Adaptive increase in sample size when interim results are promising: a practical guide with examples}, Stat in Medicine

\item FDA draft guidance (2010) \emph{Adaptive design clinical trials for drugs and biologics}

\item Anderson (2014) R package \emph{gsDesign: Group Sequential Design}

\end{itemize}
	
\end{frame}

%----------- back up slide --------------------------------------------------%
\appendix
\begin{frame}
	\frametitle{Adaptive Dose-Finding Design}
	\begin{figure}
		\begin{center}
			\includegraphics[width=1.0\textwidth,height=0.75\textheight,keepaspectratio]{3488-PB5-R1.png}
			\footnote[]{Figure 3. of Braun (2014) Chinese Clinical Oncology}
		\end{center}
	\end{figure}
\end{frame}

%----------- slide --------------------------------------------------%
\begin{frame}
	\frametitle{Adaptive Randomization Design}
	\begin{figure}
		\begin{center}
			\includegraphics[width=0.9\textwidth,height=0.75\textheight,keepaspectratio]{4210-PB4-R1.png}
			\footnote[]{Figure 2. of Zang (2014) Chinese Clinical Oncology}
		\end{center}
	\end{figure}	
\end{frame}

%----------- slide --------------------------------------------------%
\begin{frame}
	\frametitle{Drop-Loser Design}
	\begin{figure}
		\begin{center}
			\includegraphics[width=0.9\textwidth,height=0.75\textheight,keepaspectratio]{seamless-phase-II-phase-III-v1_EN.png}
			\footnote[]{Figure 3. of EUPATI (2015)}
		\end{center}
	\end{figure}	
\end{frame}
%----------- slide --------------------------------------------------%
\begin{frame}
	\frametitle{Biomarker-Adaptive Design}
	\begin{figure}
		\begin{center}
			\includegraphics[width=0.9\textwidth,height=0.7\textheight,keepaspectratio]{nrd3651-i3.jpg}
			\footnote[]{Figure 5. of Kelloff (2012)}
		\end{center}
	\end{figure}	
\end{frame}
%----------- slide --------------------------------------------------%
\begin{frame}
	\frametitle{Bayesian Design}
	\begin{figure}
		\begin{center}
			\includegraphics[width=1.0\textwidth,height=1.2\textheight,keepaspectratio]{beyond-traditional-designs-in-early-drug-development-5-728.jpg}
			\footnote[]{Spiegelhalter (1999)}
		\end{center}
	\end{figure}
\end{frame}
%----------- slide --------------------------------------------------%
%\nocite{*}
%\begin{frame}[allowframebreaks]{Key References}
%\def\newblock{\hskip .11em plus .33em minus .07em}
%\bibliographystyle{plain}
%\bibliographystyle{asa}
%\bibliography{I:/consult/research/reference/latex/ctd20111010}
%\bibliography{ctd20111010}
%\end{frame}
\end{document}

