\documentclass[]{article}
\usepackage{lmodern}
\usepackage{amssymb,amsmath}
\usepackage{ifxetex,ifluatex}
\usepackage{fixltx2e} % provides \textsubscript
\ifnum 0\ifxetex 1\fi\ifluatex 1\fi=0 % if pdftex
  \usepackage[T1]{fontenc}
  \usepackage[utf8]{inputenc}
\else % if luatex or xelatex
  \ifxetex
    \usepackage{mathspec}
  \else
    \usepackage{fontspec}
  \fi
  \defaultfontfeatures{Ligatures=TeX,Scale=MatchLowercase}
  \newcommand{\euro}{€}
\fi
% use upquote if available, for straight quotes in verbatim environments
\IfFileExists{upquote.sty}{\usepackage{upquote}}{}
% use microtype if available
\IfFileExists{microtype.sty}{%
\usepackage{microtype}
\UseMicrotypeSet[protrusion]{basicmath} % disable protrusion for tt fonts
}{}
\usepackage[margin=1in]{geometry}
\usepackage{hyperref}
\PassOptionsToPackage{usenames,dvipsnames}{color} % color is loaded by hyperref
\hypersetup{unicode=true,
            pdftitle={Bayesian design using Beta-binomial model for single-arm clinical trials},
            pdfauthor={Yalin Zhu},
            pdfborder={0 0 0},
            breaklinks=true}
\urlstyle{same}  % don't use monospace font for urls
\usepackage{color}
\usepackage{fancyvrb}
\newcommand{\VerbBar}{|}
\newcommand{\VERB}{\Verb[commandchars=\\\{\}]}
\DefineVerbatimEnvironment{Highlighting}{Verbatim}{commandchars=\\\{\}}
% Add ',fontsize=\small' for more characters per line
\usepackage{framed}
\definecolor{shadecolor}{RGB}{248,248,248}
\newenvironment{Shaded}{\begin{snugshade}}{\end{snugshade}}
\newcommand{\KeywordTok}[1]{\textcolor[rgb]{0.13,0.29,0.53}{\textbf{{#1}}}}
\newcommand{\DataTypeTok}[1]{\textcolor[rgb]{0.13,0.29,0.53}{{#1}}}
\newcommand{\DecValTok}[1]{\textcolor[rgb]{0.00,0.00,0.81}{{#1}}}
\newcommand{\BaseNTok}[1]{\textcolor[rgb]{0.00,0.00,0.81}{{#1}}}
\newcommand{\FloatTok}[1]{\textcolor[rgb]{0.00,0.00,0.81}{{#1}}}
\newcommand{\ConstantTok}[1]{\textcolor[rgb]{0.00,0.00,0.00}{{#1}}}
\newcommand{\CharTok}[1]{\textcolor[rgb]{0.31,0.60,0.02}{{#1}}}
\newcommand{\SpecialCharTok}[1]{\textcolor[rgb]{0.00,0.00,0.00}{{#1}}}
\newcommand{\StringTok}[1]{\textcolor[rgb]{0.31,0.60,0.02}{{#1}}}
\newcommand{\VerbatimStringTok}[1]{\textcolor[rgb]{0.31,0.60,0.02}{{#1}}}
\newcommand{\SpecialStringTok}[1]{\textcolor[rgb]{0.31,0.60,0.02}{{#1}}}
\newcommand{\ImportTok}[1]{{#1}}
\newcommand{\CommentTok}[1]{\textcolor[rgb]{0.56,0.35,0.01}{\textit{{#1}}}}
\newcommand{\DocumentationTok}[1]{\textcolor[rgb]{0.56,0.35,0.01}{\textbf{\textit{{#1}}}}}
\newcommand{\AnnotationTok}[1]{\textcolor[rgb]{0.56,0.35,0.01}{\textbf{\textit{{#1}}}}}
\newcommand{\CommentVarTok}[1]{\textcolor[rgb]{0.56,0.35,0.01}{\textbf{\textit{{#1}}}}}
\newcommand{\OtherTok}[1]{\textcolor[rgb]{0.56,0.35,0.01}{{#1}}}
\newcommand{\FunctionTok}[1]{\textcolor[rgb]{0.00,0.00,0.00}{{#1}}}
\newcommand{\VariableTok}[1]{\textcolor[rgb]{0.00,0.00,0.00}{{#1}}}
\newcommand{\ControlFlowTok}[1]{\textcolor[rgb]{0.13,0.29,0.53}{\textbf{{#1}}}}
\newcommand{\OperatorTok}[1]{\textcolor[rgb]{0.81,0.36,0.00}{\textbf{{#1}}}}
\newcommand{\BuiltInTok}[1]{{#1}}
\newcommand{\ExtensionTok}[1]{{#1}}
\newcommand{\PreprocessorTok}[1]{\textcolor[rgb]{0.56,0.35,0.01}{\textit{{#1}}}}
\newcommand{\AttributeTok}[1]{\textcolor[rgb]{0.77,0.63,0.00}{{#1}}}
\newcommand{\RegionMarkerTok}[1]{{#1}}
\newcommand{\InformationTok}[1]{\textcolor[rgb]{0.56,0.35,0.01}{\textbf{\textit{{#1}}}}}
\newcommand{\WarningTok}[1]{\textcolor[rgb]{0.56,0.35,0.01}{\textbf{\textit{{#1}}}}}
\newcommand{\AlertTok}[1]{\textcolor[rgb]{0.94,0.16,0.16}{{#1}}}
\newcommand{\ErrorTok}[1]{\textcolor[rgb]{0.64,0.00,0.00}{\textbf{{#1}}}}
\newcommand{\NormalTok}[1]{{#1}}
\usepackage{graphicx,grffile}
\makeatletter
\def\maxwidth{\ifdim\Gin@nat@width>\linewidth\linewidth\else\Gin@nat@width\fi}
\def\maxheight{\ifdim\Gin@nat@height>\textheight\textheight\else\Gin@nat@height\fi}
\makeatother
% Scale images if necessary, so that they will not overflow the page
% margins by default, and it is still possible to overwrite the defaults
% using explicit options in \includegraphics[width, height, ...]{}
\setkeys{Gin}{width=\maxwidth,height=\maxheight,keepaspectratio}
\setlength{\parindent}{0pt}
\setlength{\parskip}{6pt plus 2pt minus 1pt}
\setlength{\emergencystretch}{3em}  % prevent overfull lines
\providecommand{\tightlist}{%
  \setlength{\itemsep}{0pt}\setlength{\parskip}{0pt}}
\setcounter{secnumdepth}{5}

%%% Use protect on footnotes to avoid problems with footnotes in titles
\let\rmarkdownfootnote\footnote%
\def\footnote{\protect\rmarkdownfootnote}

%%% Change title format to be more compact
\usepackage{titling}

% Create subtitle command for use in maketitle
\newcommand{\subtitle}[1]{
  \posttitle{
    \begin{center}\large#1\end{center}
    }
}

\setlength{\droptitle}{-2em}
  \title{Bayesian design using Beta-binomial model for single-arm clinical trials}
  \pretitle{\vspace{\droptitle}\centering\huge}
  \posttitle{\par}
  \author{Yalin Zhu}
  \preauthor{\centering\large\emph}
  \postauthor{\par}
  \predate{\centering\large\emph}
  \postdate{\par}
  \date{July 25, 2016}


\usepackage{animate}

% Redefines (sub)paragraphs to behave more like sections
\ifx\paragraph\undefined\else
\let\oldparagraph\paragraph
\renewcommand{\paragraph}[1]{\oldparagraph{#1}\mbox{}}
\fi
\ifx\subparagraph\undefined\else
\let\oldsubparagraph\subparagraph
\renewcommand{\subparagraph}[1]{\oldsubparagraph{#1}\mbox{}}
\fi

\begin{document}
\maketitle

\section{Description}\label{description}

The purpose of a phase I trial is to study the drug's toxicity in humans
and to identify the `best' dose to be used (this is usually the highest
dose which does not result in excessive toxicity). Following this, a
phase II trial using this `best' dose is then conducted. The goal in
such a trial is to assess the effacacy of the drug (often demonstrated
by using tumour response as the indicator) to determine if it should be
further tested in a large-scale randomized phase III trial.

Phase II clinical trials thus play an important role in the development
and testing of a new drug. The main goal of a phase II trial is not to
obtain a precise estimate of the response rate of the new drug, but
rather to accept or reject the drug for further testing in a phase III
trial. A commonly used primary endpoint in phase II cancer clinical
trials is the clinical response to a treatment, which is a binary
endpoint defined as the patient achieving complete or partial response
within a predefined treatment course. In the early phase II development
of new drugs, most trials are open label, single-arm studies, while late
phase II trials tend to be multiarm, randomized studies.

Nevertheless, the analysis of the trial results typically include the
obtaining of an estimate of the true response proportion, along with the
associated 95\% (frequentist) confidence interval. Such an analysis does
not always answer the question of interest to the investigator. For
example, the investigator might wish to know the probability that the
true response proportion exceeds the prespecified target value, or he
may wish to identify the (credible) interval that has a 95\% probability
of containing the true response proportion (this is not the same as the
95\% frequentist confidence interval). Such questions can be answered
using a Bayesian approach. With a Bayesian approach, we can obtain the
posterior probability distribution of the true response proportion. This
allows us to compute the probability that the response proportion falls
within any prespecified region of interest, including the region above
the target proportion. The bayesian credible interval is the interval
that has a 95\% probability of containing the true response proportion.
A Bayesian design also allows for the formal incorporation of relevant
information from other sources of evidence in the monitoring and
analysis of the trial.

In this project, we use Beta-binomial conjugate to develop some bayesian
deisgn methods. First of all, we explore what the Bayesian prior and
posterior distribution looks like. Then we explore single-arm design
methods using posterior probability and predictive probability. Some R
functions and web applications are developed as well.

\section{Prior Elicitation and Posterior
Construction}\label{prior-elicitation-and-posterior-construction}

A strength of Bayesian design and analysis is the ability to formally
incorporate available information. But choosing a prior distribution
requires careful consideration and work. This aspect of the Bayesian
approach is more art than science. Several meetings between clinical
investigators and statisticians may be necessary for assembling,
evaluating, and quantifying the evidence based on literature or prior
experience. In the process of selecting a prior distribution, the
statistician should evaluate its sensitivity on the design's operating
characteristics. Using a non-informative prior may be appropriate. Such
a prior imitates a frequentist approach at the analysis stage but does
not take existing information into consideration. And because a
non-informative prior is artificial, it can lead to a poor design by
overreacting to early results. When incorporating historical information
into the prior, we almost always down-weight it in comparison to data
collected in the actual trial, as described above.

For a phase II single-arm trial, suppose our goal is to evaluate the
\emph{response rate} \(p\) for a new drug by testing the hypothesis
\(H_0: p \le p_0\) versus \(H_1: p \ge p_1 = p_0 + \delta\), which
implies \(p>p_0\). We assume that the prior distribution of the response
rate follows a Beta distribution,

\[p \sim Beta(a, b).\]

In the Bayesian methods, the \(p\) is regarded as a random variable
(which is fixed parameter for frequentist). The quantity
\(\dfrac{a}{a + b}\) and \(\dfrac{ab}{(a+b)^2(a+b+1)}\) gives the prior
mean and variance, while the magnitude of \(a + b\) indicates how
informative the prior is. Since the quantities \textbf{\(a\)} and
\textbf{\(b\)} can be considered as the numbers of effective prior
\textbf{responses} and \textbf{non-responses}, respectively, \(a+b\) can
be thought of as a measure of prior precision: the larger this sum, the
more informative the prior and the stronger the belief it contains.

\subsection{Choose the prior distribution and
parameters}\label{choose-the-prior-distribution-and-parameters}

Let us look into the beta-binomial bayesian frame. First of all, we need
to select parameter of \(Beta(a,b)\). There are several methods to
choose the prior parameter in the exsiting literatures.

\subsubsection{Simply choose a non-inormative
prior.}\label{simply-choose-a-non-inormative-prior.}

For example, \(Beta(1,1)\) (equivalent to \$U niform(0,1)), since this
kind of prior provides very little information, it is also called vague
prior.

We can look at how the posterior proabability changes with more patients
enrolled (prior updated) under the \(Beta(1,1)\) prior.

\subsubsection{Plot the prior, likelihood and posterior density
functions}\label{plot-the-prior-likelihood-and-posterior-density-functions}

We develop an R function which not only plots the posterior tendency
with updating previous posterior as a new prior, but also provide a full
list of inference information (indlucing outcome cohort data, posterior
mean, credible interval)

\begin{Shaded}
\begin{Highlighting}[]
\KeywordTok{source}\NormalTok{(}\StringTok{"animation_update.R"}\NormalTok{)}
\end{Highlighting}
\end{Shaded}

After CSCC patients receiving the cancer treatment neo-adjuvant therapy
and surgery, consider the single primary endpoint: pathological complete
response (\(pCR\)) with the following hypotheses:
\[H_0: pCR \le 15\% \quad versus \quad H_1: pCR > 15\%\] If the study
sequtially monitors the prior and posteror, we can plot the
distributions and likelihood as animations with the simulated data.

\begin{Shaded}
\begin{Highlighting}[]
\KeywordTok{BB.sim}\NormalTok{(}\DataTypeTok{M=}\DecValTok{20}\NormalTok{,}\DataTypeTok{N=}\DecValTok{1}\NormalTok{,}\DataTypeTok{p=}\FloatTok{0.15}\NormalTok{)}
\end{Highlighting}
\end{Shaded}

\begin{verbatim}
## Prior: Beta(1,1) 
## 
## ======== Cohort Number: 1 ======== 
## Observations -- Sample Size: 1(1)  ||  Number of Response: 0(0)  ||  Number of Failure: 1(1)
##   Observed Response Rate: 0
## Posterior: Beta(1,2) 
##   Posterior Mean: 0.333,  Difference between posterior and true response rate: 0.333
##   95% Credible Interval: (0.0126,0.842)
\end{verbatim}

\begin{verbatim}
## ======== Cohort Number: 2 ======== 
## Observations -- Sample Size: 2(1)  ||  Number of Response: 0(0)  ||  Number of Failure: 2(1)
##   Observed Response Rate: 0
## Posterior: Beta(1,3) 
##   Posterior Mean: 0.25,  Difference between posterior and true response rate: 0.25
##   95% Credible Interval: (0.0084,0.708)
\end{verbatim}

\begin{verbatim}
## ======== Cohort Number: 3 ======== 
## Observations -- Sample Size: 3(1)  ||  Number of Response: 0(0)  ||  Number of Failure: 3(1)
##   Observed Response Rate: 0
## Posterior: Beta(1,4) 
##   Posterior Mean: 0.2,  Difference between posterior and true response rate: 0.2
##   95% Credible Interval: (0.00631,0.602)
\end{verbatim}

\begin{verbatim}
## ======== Cohort Number: 4 ======== 
## Observations -- Sample Size: 4(1)  ||  Number of Response: 1(1)  ||  Number of Failure: 3(0)
##   Observed Response Rate: 0.25
## Posterior: Beta(2,4) 
##   Posterior Mean: 0.333,  Difference between posterior and true response rate: 0.0833
##   95% Credible Interval: (0.0527,0.716)
\end{verbatim}

\begin{verbatim}
## ======== Cohort Number: 5 ======== 
## Observations -- Sample Size: 5(1)  ||  Number of Response: 2(1)  ||  Number of Failure: 3(0)
##   Observed Response Rate: 0.4
## Posterior: Beta(3,4) 
##   Posterior Mean: 0.429,  Difference between posterior and true response rate: 0.0286
##   95% Credible Interval: (0.118,0.777)
\end{verbatim}

\begin{verbatim}
## ======== Cohort Number: 6 ======== 
## Observations -- Sample Size: 6(1)  ||  Number of Response: 2(0)  ||  Number of Failure: 4(1)
##   Observed Response Rate: 0.333
## Posterior: Beta(3,5) 
##   Posterior Mean: 0.375,  Difference between posterior and true response rate: 0.0417
##   95% Credible Interval: (0.099,0.71)
\end{verbatim}

\begin{verbatim}
## ======== Cohort Number: 7 ======== 
## Observations -- Sample Size: 7(1)  ||  Number of Response: 2(0)  ||  Number of Failure: 5(1)
##   Observed Response Rate: 0.286
## Posterior: Beta(3,6) 
##   Posterior Mean: 0.333,  Difference between posterior and true response rate: 0.0476
##   95% Credible Interval: (0.0852,0.651)
\end{verbatim}

\begin{verbatim}
## ======== Cohort Number: 8 ======== 
## Observations -- Sample Size: 8(1)  ||  Number of Response: 3(1)  ||  Number of Failure: 5(0)
##   Observed Response Rate: 0.375
## Posterior: Beta(4,6) 
##   Posterior Mean: 0.4,  Difference between posterior and true response rate: 0.025
##   95% Credible Interval: (0.137,0.701)
\end{verbatim}

\begin{verbatim}
## ======== Cohort Number: 9 ======== 
## Observations -- Sample Size: 9(1)  ||  Number of Response: 3(0)  ||  Number of Failure: 6(1)
##   Observed Response Rate: 0.333
## Posterior: Beta(4,7) 
##   Posterior Mean: 0.364,  Difference between posterior and true response rate: 0.0303
##   95% Credible Interval: (0.122,0.652)
\end{verbatim}

\begin{verbatim}
## ======== Cohort Number: 10 ======== 
## Observations -- Sample Size: 10(1)  ||  Number of Response: 3(0)  ||  Number of Failure: 7(1)
##   Observed Response Rate: 0.3
## Posterior: Beta(4,8) 
##   Posterior Mean: 0.333,  Difference between posterior and true response rate: 0.0333
##   95% Credible Interval: (0.109,0.61)
\end{verbatim}

\begin{verbatim}
## ======== Cohort Number: 11 ======== 
## Observations -- Sample Size: 11(1)  ||  Number of Response: 4(1)  ||  Number of Failure: 7(0)
##   Observed Response Rate: 0.364
## Posterior: Beta(5,8) 
##   Posterior Mean: 0.385,  Difference between posterior and true response rate: 0.021
##   95% Credible Interval: (0.152,0.651)
\end{verbatim}

\begin{verbatim}
## ======== Cohort Number: 12 ======== 
## Observations -- Sample Size: 12(1)  ||  Number of Response: 4(0)  ||  Number of Failure: 8(1)
##   Observed Response Rate: 0.333
## Posterior: Beta(5,9) 
##   Posterior Mean: 0.357,  Difference between posterior and true response rate: 0.0238
##   95% Credible Interval: (0.139,0.614)
\end{verbatim}

\begin{verbatim}
## ======== Cohort Number: 13 ======== 
## Observations -- Sample Size: 13(1)  ||  Number of Response: 4(0)  ||  Number of Failure: 9(1)
##   Observed Response Rate: 0.308
## Posterior: Beta(5,10) 
##   Posterior Mean: 0.333,  Difference between posterior and true response rate: 0.0256
##   95% Credible Interval: (0.128,0.581)
\end{verbatim}

\begin{verbatim}
## ======== Cohort Number: 14 ======== 
## Observations -- Sample Size: 14(1)  ||  Number of Response: 4(0)  ||  Number of Failure: 10(1)
##   Observed Response Rate: 0.286
## Posterior: Beta(5,11) 
##   Posterior Mean: 0.312,  Difference between posterior and true response rate: 0.0268
##   95% Credible Interval: (0.118,0.551)
\end{verbatim}

\begin{verbatim}
## ======== Cohort Number: 15 ======== 
## Observations -- Sample Size: 15(1)  ||  Number of Response: 4(0)  ||  Number of Failure: 11(1)
##   Observed Response Rate: 0.267
## Posterior: Beta(5,12) 
##   Posterior Mean: 0.294,  Difference between posterior and true response rate: 0.0275
##   95% Credible Interval: (0.11,0.524)
\end{verbatim}

\begin{verbatim}
## ======== Cohort Number: 16 ======== 
## Observations -- Sample Size: 16(1)  ||  Number of Response: 5(1)  ||  Number of Failure: 11(0)
##   Observed Response Rate: 0.312
## Posterior: Beta(6,12) 
##   Posterior Mean: 0.333,  Difference between posterior and true response rate: 0.0208
##   95% Credible Interval: (0.142,0.56)
\end{verbatim}

\begin{verbatim}
## ======== Cohort Number: 17 ======== 
## Observations -- Sample Size: 17(1)  ||  Number of Response: 5(0)  ||  Number of Failure: 12(1)
##   Observed Response Rate: 0.294
## Posterior: Beta(6,13) 
##   Posterior Mean: 0.316,  Difference between posterior and true response rate: 0.0217
##   95% Credible Interval: (0.133,0.535)
\end{verbatim}

\begin{verbatim}
## ======== Cohort Number: 18 ======== 
## Observations -- Sample Size: 18(1)  ||  Number of Response: 5(0)  ||  Number of Failure: 13(1)
##   Observed Response Rate: 0.278
## Posterior: Beta(6,14) 
##   Posterior Mean: 0.3,  Difference between posterior and true response rate: 0.0222
##   95% Credible Interval: (0.126,0.512)
\end{verbatim}

\begin{verbatim}
## ======== Cohort Number: 19 ======== 
## Observations -- Sample Size: 19(1)  ||  Number of Response: 5(0)  ||  Number of Failure: 14(1)
##   Observed Response Rate: 0.263
## Posterior: Beta(6,15) 
##   Posterior Mean: 0.286,  Difference between posterior and true response rate: 0.0226
##   95% Credible Interval: (0.119,0.491)
\end{verbatim}

\begin{verbatim}
## ======== Cohort Number: 20 ======== 
## Observations -- Sample Size: 20(1)  ||  Number of Response: 6(1)  ||  Number of Failure: 14(0)
##   Observed Response Rate: 0.3
## Posterior: Beta(7,15) 
##   Posterior Mean: 0.318,  Difference between posterior and true response rate: 0.0182
##   95% Credible Interval: (0.146,0.522)
\end{verbatim}

\animategraphics[width=7in,controls,loop]{1}{vignette-BetaBinomial-pdf_files/figure-latex/unnamed-chunk-2-}{1}{20}

We can observe after 10 patients enrolled, the difference between
posterior and true response rate reduced to a stable level below 3\%.

We can also monitor the patients by cohort. Simulate each cohort
contains 5 patients, the results are shown as follows:

\begin{Shaded}
\begin{Highlighting}[]
\KeywordTok{BB.sim}\NormalTok{(}\DataTypeTok{M=}\DecValTok{10}\NormalTok{,}\DataTypeTok{N=}\DecValTok{5}\NormalTok{,}\DataTypeTok{p=}\FloatTok{0.15}\NormalTok{)}
\end{Highlighting}
\end{Shaded}

\begin{verbatim}
## Prior: Beta(1,1) 
## 
## ======== Cohort Number: 1 ======== 
## Observations -- Sample Size: 5(5)  ||  Number of Response: 0(0)  ||  Number of Failure: 5(5)
##   Observed Response Rate: 0
## Posterior: Beta(1,6) 
##   Posterior Mean: 0.143,  Difference between posterior and true response rate: 0.143
##   95% Credible Interval: (0.00421,0.459)
\end{verbatim}

\begin{verbatim}
## ======== Cohort Number: 2 ======== 
## Observations -- Sample Size: 10(5)  ||  Number of Response: 1(1)  ||  Number of Failure: 9(4)
##   Observed Response Rate: 0.1
## Posterior: Beta(2,10) 
##   Posterior Mean: 0.167,  Difference between posterior and true response rate: 0.0667
##   95% Credible Interval: (0.0228,0.413)
\end{verbatim}

\begin{verbatim}
## ======== Cohort Number: 3 ======== 
## Observations -- Sample Size: 15(5)  ||  Number of Response: 1(0)  ||  Number of Failure: 14(5)
##   Observed Response Rate: 0.0667
## Posterior: Beta(2,15) 
##   Posterior Mean: 0.118,  Difference between posterior and true response rate: 0.051
##   95% Credible Interval: (0.0155,0.302)
\end{verbatim}

\begin{verbatim}
## ======== Cohort Number: 4 ======== 
## Observations -- Sample Size: 20(5)  ||  Number of Response: 3(2)  ||  Number of Failure: 17(3)
##   Observed Response Rate: 0.15
## Posterior: Beta(4,18) 
##   Posterior Mean: 0.182,  Difference between posterior and true response rate: 0.0318
##   95% Credible Interval: (0.0545,0.363)
\end{verbatim}

\begin{verbatim}
## ======== Cohort Number: 5 ======== 
## Observations -- Sample Size: 25(5)  ||  Number of Response: 5(2)  ||  Number of Failure: 20(3)
##   Observed Response Rate: 0.2
## Posterior: Beta(6,21) 
##   Posterior Mean: 0.222,  Difference between posterior and true response rate: 0.0222
##   95% Credible Interval: (0.0897,0.394)
\end{verbatim}

\begin{verbatim}
## ======== Cohort Number: 6 ======== 
## Observations -- Sample Size: 30(5)  ||  Number of Response: 5(0)  ||  Number of Failure: 25(5)
##   Observed Response Rate: 0.167
## Posterior: Beta(6,26) 
##   Posterior Mean: 0.188,  Difference between posterior and true response rate: 0.0208
##   95% Credible Interval: (0.0745,0.337)
\end{verbatim}

\begin{verbatim}
## ======== Cohort Number: 7 ======== 
## Observations -- Sample Size: 35(5)  ||  Number of Response: 6(1)  ||  Number of Failure: 29(4)
##   Observed Response Rate: 0.171
## Posterior: Beta(7,30) 
##   Posterior Mean: 0.189,  Difference between posterior and true response rate: 0.0178
##   95% Credible Interval: (0.0819,0.328)
\end{verbatim}

\begin{verbatim}
## ======== Cohort Number: 8 ======== 
## Observations -- Sample Size: 40(5)  ||  Number of Response: 8(2)  ||  Number of Failure: 32(3)
##   Observed Response Rate: 0.2
## Posterior: Beta(9,33) 
##   Posterior Mean: 0.214,  Difference between posterior and true response rate: 0.0143
##   95% Credible Interval: (0.106,0.349)
\end{verbatim}

\begin{verbatim}
## ======== Cohort Number: 9 ======== 
## Observations -- Sample Size: 45(5)  ||  Number of Response: 9(1)  ||  Number of Failure: 36(4)
##   Observed Response Rate: 0.2
## Posterior: Beta(10,37) 
##   Posterior Mean: 0.213,  Difference between posterior and true response rate: 0.0128
##   95% Credible Interval: (0.109,0.339)
\end{verbatim}

\begin{verbatim}
## ======== Cohort Number: 10 ======== 
## Observations -- Sample Size: 50(5)  ||  Number of Response: 10(1)  ||  Number of Failure: 40(4)
##   Observed Response Rate: 0.2
## Posterior: Beta(11,41) 
##   Posterior Mean: 0.212,  Difference between posterior and true response rate: 0.0115
##   95% Credible Interval: (0.113,0.331)
\end{verbatim}

\animategraphics[width=7in,controls,loop]{1}{vignette-BetaBinomial-pdf_files/figure-latex/unnamed-chunk-3-}{1}{10}

After only 4 cohort (20 patients) enrolled, the difference between
posterior and true response rate reduced to a stable level below 3\%.

\subsubsection{Mean and Variance prior}\label{mean-and-variance-prior}

Based on the mean \(\mu\) and the variance \(\sigma^2\), we can derive
the prior parameter of \(Beta(a,b)\) distribution with
\[a=\mu\left\{\dfrac{\mu(1-\mu)}{\sigma^2}-1\right\}\] and
\[b=(1-\mu)\left\{\dfrac{\mu(1-\mu)}{\sigma^2}-1\right\}.\]

\textbf{Notations}

\(\mu_0\): mean of prior response probability;

\(\sigma_0\): standard deviation of prior response probability;

\(n\): total sample size;

\(x\): number of response subjects;

For the mean and variance prior, We reproduce the arsenic trioxide
trials example (Zohar, Teramukai, and Zhou 2008), with MM and APL data.
The results are shown as follows.

\begin{Shaded}
\begin{Highlighting}[]
\KeywordTok{source}\NormalTok{(}\StringTok{"simple_design.R"}\NormalTok{)}
\end{Highlighting}
\end{Shaded}

\begin{Shaded}
\begin{Highlighting}[]
\NormalTok{## create MM data and set the prior mean and variance}
\NormalTok{MM.r =}\StringTok{ }\KeywordTok{rep}\NormalTok{(}\DecValTok{0}\NormalTok{, }\DecValTok{12}\NormalTok{)}
\NormalTok{MM.mean =}\StringTok{ }\FloatTok{0.1}
\NormalTok{MM.var =}\StringTok{ }\FloatTok{0.0225}
\KeywordTok{post.mean}\NormalTok{(}\DataTypeTok{mu0 =} \NormalTok{MM.mean, }\DataTypeTok{sigma0 =} \KeywordTok{sqrt}\NormalTok{(MM.var), }\DataTypeTok{r =} \NormalTok{MM.r)}
\end{Highlighting}
\end{Shaded}

\begin{verbatim}
## $para.a
## [1] 0.3
## 
## $para.b
## [1] 2.7
## 
## $`poterior mean`
##  [1] 0.0750 0.0600 0.0500 0.0429 0.0375 0.0333 0.0300 0.0273 0.0250 0.0231
## [11] 0.0214 0.0200
\end{verbatim}

\begin{Shaded}
\begin{Highlighting}[]
\NormalTok{## create APL data and set the prior mean and variance}
\NormalTok{APL.r <-}\StringTok{ }\KeywordTok{c}\NormalTok{(}\DecValTok{0}\NormalTok{, }\DecValTok{1}\NormalTok{, }\DecValTok{0}\NormalTok{, }\DecValTok{0}\NormalTok{, }\DecValTok{1}\NormalTok{, }\DecValTok{1}\NormalTok{, }\DecValTok{1}\NormalTok{, }\DecValTok{1}\NormalTok{, }\DecValTok{0}\NormalTok{, }\DecValTok{1}\NormalTok{, }\DecValTok{1}\NormalTok{, }\DecValTok{1}\NormalTok{, }\DecValTok{0}\NormalTok{, }\DecValTok{1}\NormalTok{, }\DecValTok{1}\NormalTok{, }\DecValTok{1}\NormalTok{, }\DecValTok{1}\NormalTok{, }\DecValTok{1}\NormalTok{, }\DecValTok{1}\NormalTok{, }\DecValTok{1}\NormalTok{)}
\NormalTok{APL.mean =}\StringTok{ }\FloatTok{0.3}
\NormalTok{APL.var =}\StringTok{ }\FloatTok{0.0191}
\KeywordTok{post.mean}\NormalTok{(}\DataTypeTok{mu0 =} \NormalTok{APL.mean, }\DataTypeTok{sigma0 =} \KeywordTok{sqrt}\NormalTok{(APL.var), }\DataTypeTok{r =} \NormalTok{APL.r)}
\end{Highlighting}
\end{Shaded}

\begin{verbatim}
## $para.a
## [1] 3
## 
## $para.b
## [1] 7
## 
## $`poterior mean`
##  [1] 0.273 0.333 0.308 0.286 0.333 0.375 0.412 0.444 0.421 0.450 0.476
## [12] 0.500 0.478 0.500 0.520 0.539 0.556 0.571 0.586 0.600
\end{verbatim}

We also create R function to plot animations for the MM and APL data

\begin{Shaded}
\begin{Highlighting}[]
\NormalTok{MM.r <-}\StringTok{ }\KeywordTok{rep}\NormalTok{(}\DecValTok{0}\NormalTok{,}\DecValTok{12}\NormalTok{)}
\KeywordTok{BB.plot}\NormalTok{(}\FloatTok{0.3}\NormalTok{,}\FloatTok{2.7}\NormalTok{,}\DataTypeTok{r=}\NormalTok{MM.r)}
\end{Highlighting}
\end{Shaded}

\begin{verbatim}
## Prior: Beta(0.3,2.7) 
## 
## ======== Cohort Number: 1 ======== 
## Observations -- Sample Size: 1(1)  ||  Number of Response: 0(0)  ||  Number of Failure: 1(1)
##   Observed Response Rate: 0
## Posterior: Beta(0.3,3.7) 
##   Posterior Mean: 0.075
##   95% Credible Interval: (9.48e-07,0.43)
\end{verbatim}

\begin{verbatim}
## ======== Cohort Number: 2 ======== 
## Observations -- Sample Size: 2(1)  ||  Number of Response: 0(0)  ||  Number of Failure: 2(1)
##   Observed Response Rate: 0
## Posterior: Beta(0.3,4.7) 
##   Posterior Mean: 0.06
##   95% Credible Interval: (7.31e-07,0.353)
\end{verbatim}

\begin{verbatim}
## ======== Cohort Number: 3 ======== 
## Observations -- Sample Size: 3(1)  ||  Number of Response: 0(0)  ||  Number of Failure: 3(1)
##   Observed Response Rate: 0
## Posterior: Beta(0.3,5.7) 
##   Posterior Mean: 0.05
##   95% Credible Interval: (5.95e-07,0.298)
\end{verbatim}

\begin{verbatim}
## ======== Cohort Number: 4 ======== 
## Observations -- Sample Size: 4(1)  ||  Number of Response: 0(0)  ||  Number of Failure: 4(1)
##   Observed Response Rate: 0
## Posterior: Beta(0.3,6.7) 
##   Posterior Mean: 0.0429
##   95% Credible Interval: (5.01e-07,0.258)
\end{verbatim}

\begin{verbatim}
## ======== Cohort Number: 5 ======== 
## Observations -- Sample Size: 5(1)  ||  Number of Response: 0(0)  ||  Number of Failure: 5(1)
##   Observed Response Rate: 0
## Posterior: Beta(0.3,7.7) 
##   Posterior Mean: 0.0375
##   95% Credible Interval: (4.33e-07,0.227)
\end{verbatim}

\begin{verbatim}
## ======== Cohort Number: 6 ======== 
## Observations -- Sample Size: 6(1)  ||  Number of Response: 0(0)  ||  Number of Failure: 6(1)
##   Observed Response Rate: 0
## Posterior: Beta(0.3,8.7) 
##   Posterior Mean: 0.0333
##   95% Credible Interval: (3.81e-07,0.203)
\end{verbatim}

\begin{verbatim}
## ======== Cohort Number: 7 ======== 
## Observations -- Sample Size: 7(1)  ||  Number of Response: 0(0)  ||  Number of Failure: 7(1)
##   Observed Response Rate: 0
## Posterior: Beta(0.3,9.7) 
##   Posterior Mean: 0.03
##   95% Credible Interval: (3.41e-07,0.184)
\end{verbatim}

\begin{verbatim}
## ======== Cohort Number: 8 ======== 
## Observations -- Sample Size: 8(1)  ||  Number of Response: 0(0)  ||  Number of Failure: 8(1)
##   Observed Response Rate: 0
## Posterior: Beta(0.3,10.7) 
##   Posterior Mean: 0.0273
##   95% Credible Interval: (3.08e-07,0.168)
\end{verbatim}

\begin{verbatim}
## ======== Cohort Number: 9 ======== 
## Observations -- Sample Size: 9(1)  ||  Number of Response: 0(0)  ||  Number of Failure: 9(1)
##   Observed Response Rate: 0
## Posterior: Beta(0.3,11.7) 
##   Posterior Mean: 0.025
##   95% Credible Interval: (2.81e-07,0.154)
\end{verbatim}

\begin{verbatim}
## ======== Cohort Number: 10 ======== 
## Observations -- Sample Size: 10(1)  ||  Number of Response: 0(0)  ||  Number of Failure: 10(1)
##   Observed Response Rate: 0
## Posterior: Beta(0.3,12.7) 
##   Posterior Mean: 0.0231
##   95% Credible Interval: (2.58e-07,0.143)
\end{verbatim}

\begin{verbatim}
## ======== Cohort Number: 11 ======== 
## Observations -- Sample Size: 11(1)  ||  Number of Response: 0(0)  ||  Number of Failure: 11(1)
##   Observed Response Rate: 0
## Posterior: Beta(0.3,13.7) 
##   Posterior Mean: 0.0214
##   95% Credible Interval: (2.39e-07,0.133)
\end{verbatim}

\begin{verbatim}
## ======== Cohort Number: 12 ======== 
## Observations -- Sample Size: 12(1)  ||  Number of Response: 0(0)  ||  Number of Failure: 12(1)
##   Observed Response Rate: 0
## Posterior: Beta(0.3,14.7) 
##   Posterior Mean: 0.02
##   95% Credible Interval: (2.22e-07,0.124)
\end{verbatim}

\animategraphics[width=7in,controls,loop]{1}{vignette-BetaBinomial-pdf_files/figure-latex/unnamed-chunk-6-}{1}{12}

\begin{Shaded}
\begin{Highlighting}[]
\NormalTok{APL.r <-}\StringTok{ }\KeywordTok{c}\NormalTok{(}\DecValTok{0}\NormalTok{,}\DecValTok{1}\NormalTok{,}\DecValTok{0}\NormalTok{,}\DecValTok{0}\NormalTok{,}\DecValTok{1}\NormalTok{,}\DecValTok{1}\NormalTok{,}\DecValTok{1}\NormalTok{,}\DecValTok{1}\NormalTok{,}\DecValTok{0}\NormalTok{,}\DecValTok{1}\NormalTok{,}\DecValTok{1}\NormalTok{,}\DecValTok{1}\NormalTok{,}\DecValTok{0}\NormalTok{,}\DecValTok{1}\NormalTok{,}\DecValTok{1}\NormalTok{,}\DecValTok{1}\NormalTok{,}\DecValTok{1}\NormalTok{,}\DecValTok{1}\NormalTok{,}\DecValTok{1}\NormalTok{,}\DecValTok{1}\NormalTok{)}
\KeywordTok{BB.plot}\NormalTok{(}\DecValTok{3}\NormalTok{,}\DecValTok{7}\NormalTok{,}\DataTypeTok{r=}\NormalTok{APL.r)}
\end{Highlighting}
\end{Shaded}

\begin{verbatim}
## Prior: Beta(3,7) 
## 
## ======== Cohort Number: 1 ======== 
## Observations -- Sample Size: 1(1)  ||  Number of Response: 0(0)  ||  Number of Failure: 1(1)
##   Observed Response Rate: 0
## Posterior: Beta(3,8) 
##   Posterior Mean: 0.273
##   95% Credible Interval: (0.0667,0.556)
\end{verbatim}

\begin{verbatim}
## ======== Cohort Number: 2 ======== 
## Observations -- Sample Size: 2(1)  ||  Number of Response: 1(1)  ||  Number of Failure: 1(0)
##   Observed Response Rate: 0.5
## Posterior: Beta(4,8) 
##   Posterior Mean: 0.333
##   95% Credible Interval: (0.109,0.61)
\end{verbatim}

\begin{verbatim}
## ======== Cohort Number: 3 ======== 
## Observations -- Sample Size: 3(1)  ||  Number of Response: 1(0)  ||  Number of Failure: 2(1)
##   Observed Response Rate: 0.333
## Posterior: Beta(4,9) 
##   Posterior Mean: 0.308
##   95% Credible Interval: (0.0992,0.572)
\end{verbatim}

\begin{verbatim}
## ======== Cohort Number: 4 ======== 
## Observations -- Sample Size: 4(1)  ||  Number of Response: 1(0)  ||  Number of Failure: 3(1)
##   Observed Response Rate: 0.25
## Posterior: Beta(4,10) 
##   Posterior Mean: 0.286
##   95% Credible Interval: (0.0909,0.538)
\end{verbatim}

\begin{verbatim}
## ======== Cohort Number: 5 ======== 
## Observations -- Sample Size: 5(1)  ||  Number of Response: 2(1)  ||  Number of Failure: 3(0)
##   Observed Response Rate: 0.4
## Posterior: Beta(5,10) 
##   Posterior Mean: 0.333
##   95% Credible Interval: (0.128,0.581)
\end{verbatim}

\begin{verbatim}
## ======== Cohort Number: 6 ======== 
## Observations -- Sample Size: 6(1)  ||  Number of Response: 3(1)  ||  Number of Failure: 3(0)
##   Observed Response Rate: 0.5
## Posterior: Beta(6,10) 
##   Posterior Mean: 0.375
##   95% Credible Interval: (0.163,0.616)
\end{verbatim}

\begin{verbatim}
## ======== Cohort Number: 7 ======== 
## Observations -- Sample Size: 7(1)  ||  Number of Response: 4(1)  ||  Number of Failure: 3(0)
##   Observed Response Rate: 0.571
## Posterior: Beta(7,10) 
##   Posterior Mean: 0.412
##   95% Credible Interval: (0.198,0.646)
\end{verbatim}

\begin{verbatim}
## ======== Cohort Number: 8 ======== 
## Observations -- Sample Size: 8(1)  ||  Number of Response: 5(1)  ||  Number of Failure: 3(0)
##   Observed Response Rate: 0.625
## Posterior: Beta(8,10) 
##   Posterior Mean: 0.444
##   95% Credible Interval: (0.23,0.671)
\end{verbatim}

\begin{verbatim}
## ======== Cohort Number: 9 ======== 
## Observations -- Sample Size: 9(1)  ||  Number of Response: 5(0)  ||  Number of Failure: 4(1)
##   Observed Response Rate: 0.556
## Posterior: Beta(8,11) 
##   Posterior Mean: 0.421
##   95% Credible Interval: (0.215,0.643)
\end{verbatim}

\begin{verbatim}
## ======== Cohort Number: 10 ======== 
## Observations -- Sample Size: 10(1)  ||  Number of Response: 6(1)  ||  Number of Failure: 4(0)
##   Observed Response Rate: 0.6
## Posterior: Beta(9,11) 
##   Posterior Mean: 0.45
##   95% Credible Interval: (0.244,0.665)
\end{verbatim}

\begin{verbatim}
## ======== Cohort Number: 11 ======== 
## Observations -- Sample Size: 11(1)  ||  Number of Response: 7(1)  ||  Number of Failure: 4(0)
##   Observed Response Rate: 0.636
## Posterior: Beta(10,11) 
##   Posterior Mean: 0.476
##   95% Credible Interval: (0.272,0.685)
\end{verbatim}

\begin{verbatim}
## ======== Cohort Number: 12 ======== 
## Observations -- Sample Size: 12(1)  ||  Number of Response: 8(1)  ||  Number of Failure: 4(0)
##   Observed Response Rate: 0.667
## Posterior: Beta(11,11) 
##   Posterior Mean: 0.5
##   95% Credible Interval: (0.298,0.702)
\end{verbatim}

\begin{verbatim}
## ======== Cohort Number: 13 ======== 
## Observations -- Sample Size: 13(1)  ||  Number of Response: 8(0)  ||  Number of Failure: 5(1)
##   Observed Response Rate: 0.615
## Posterior: Beta(11,12) 
##   Posterior Mean: 0.478
##   95% Credible Interval: (0.282,0.678)
\end{verbatim}

\begin{verbatim}
## ======== Cohort Number: 14 ======== 
## Observations -- Sample Size: 14(1)  ||  Number of Response: 9(1)  ||  Number of Failure: 5(0)
##   Observed Response Rate: 0.643
## Posterior: Beta(12,12) 
##   Posterior Mean: 0.5
##   95% Credible Interval: (0.306,0.694)
\end{verbatim}

\begin{verbatim}
## ======== Cohort Number: 15 ======== 
## Observations -- Sample Size: 15(1)  ||  Number of Response: 10(1)  ||  Number of Failure: 5(0)
##   Observed Response Rate: 0.667
## Posterior: Beta(13,12) 
##   Posterior Mean: 0.52
##   95% Credible Interval: (0.328,0.709)
\end{verbatim}

\begin{verbatim}
## ======== Cohort Number: 16 ======== 
## Observations -- Sample Size: 16(1)  ||  Number of Response: 11(1)  ||  Number of Failure: 5(0)
##   Observed Response Rate: 0.688
## Posterior: Beta(14,12) 
##   Posterior Mean: 0.538
##   95% Credible Interval: (0.349,0.722)
\end{verbatim}

\begin{verbatim}
## ======== Cohort Number: 17 ======== 
## Observations -- Sample Size: 17(1)  ||  Number of Response: 12(1)  ||  Number of Failure: 5(0)
##   Observed Response Rate: 0.706
## Posterior: Beta(15,12) 
##   Posterior Mean: 0.556
##   95% Credible Interval: (0.369,0.734)
\end{verbatim}

\begin{verbatim}
## ======== Cohort Number: 18 ======== 
## Observations -- Sample Size: 18(1)  ||  Number of Response: 13(1)  ||  Number of Failure: 5(0)
##   Observed Response Rate: 0.722
## Posterior: Beta(16,12) 
##   Posterior Mean: 0.571
##   95% Credible Interval: (0.388,0.745)
\end{verbatim}

\begin{verbatim}
## ======== Cohort Number: 19 ======== 
## Observations -- Sample Size: 19(1)  ||  Number of Response: 14(1)  ||  Number of Failure: 5(0)
##   Observed Response Rate: 0.737
## Posterior: Beta(17,12) 
##   Posterior Mean: 0.586
##   95% Credible Interval: (0.406,0.755)
\end{verbatim}

\begin{verbatim}
## ======== Cohort Number: 20 ======== 
## Observations -- Sample Size: 20(1)  ||  Number of Response: 15(1)  ||  Number of Failure: 5(0)
##   Observed Response Rate: 0.75
## Posterior: Beta(18,12) 
##   Posterior Mean: 0.6
##   95% Credible Interval: (0.423,0.765)
\end{verbatim}

\animategraphics[width=7in,controls,loop]{1}{vignette-BetaBinomial-pdf_files/figure-latex/unnamed-chunk-7-}{1}{20}

The results is identical with original paper's results.

\subsection{A Web application for choosing prior and simulating
posterior
distributions}\label{a-web-application-for-choosing-prior-and-simulating-posterior-distributions}

For users' convenience, we also develop a web application to provide the
prior information and posterior distributions inferences. Please refer
to the following link:
\url{https://allen.shinyapps.io/Beta_Bayes_Prior/}

The interface is display in the Figure 1:

\begin{figure}[htbp]
\centering
\includegraphics{demo.png}
\caption{Display of the web application.}
\end{figure}

\textbf{Remark} Credible intervals are not unique on a posterior
distribution. There are two main methods to define a suitble credible
interval:

\begin{itemize}
\tightlist
\item
  \textbf{Equal-tailed (ET)} interval is the interval where the
  probability of being below the interval is as likely as being above
  it. This interval will include the median.
\item
  \textbf{Highest Posterior Density (HPD)} interval is the narrowest
  interval, which for a unimodal distribution will involve choosing
  those values of highest probability density including the mode.
\end{itemize}

How to choose the credible interval is also our interest.

\section{Single-arm Design Studies}\label{single-arm-design-studies}

We mainly want to determin the following quantities:

\begin{itemize}
\tightlist
\item
  Sample Size Determination (\(N_{max}\))
\item
  Stopping Boundary (\(\theta_L\), \(\theta_U\))
\end{itemize}

We consider the following two bayesian methods: Posterior Probability
(abbr. ``PostP'') and Predictive Probability (abbr. ``PredP''). For
Phase IIA (earlier) trials, we prefer to allow early stopping due to
futility, but not due to efficacy, and final stop due to efficacy or
pre-spicified \(N_{max}\).

\subsection{\texorpdfstring{Sequential Stopping based on Posterior
Probability design
(\texttt{PostP})}{Sequential Stopping based on Posterior Probability design (PostP)}}\label{sequential-stopping-based-on-posterior-probability-design-postp}

A simple and practical method is to use the posterior probability to
monitor the trials. Dring each trial, the data are monitored
continuously, and decisions are made adaptively until the pre-specified
maximum sample size \(N_{max}\) is reached.(Thall and Simon 1994)

Suppose we have decieded a maximum number of accrued patients
\(N_{max}\), and assume that the number of responses \(X\) among the
current \(n\) patients (\(n \le N_{max}\)) follows a \(Binomial(n, p)\)
distribution. By the conjugacy of the beta prior and binomial
likelihood, the posterior distribution of the response rate \(p|X=x\) is
still a beta distribution,

\[p|x \sim Beta(a + x, b + n - x).\]

Then the posterior probability\\
\(PostP=Pr(p>p_0|x)\) can be used to decide the sample size and stopping
boundary.

\subsubsection{Algorithm 1}\label{algorithm-1}

\begin{itemize}
\tightlist
\item
  \textbf{Step 1:} Specified the upper and lower probability cutoffs
  \(\theta_U\) and \(\theta_L\). Typically, \(\theta_U \in [0.9,1]\) and
  \(\theta_L \in [0,0.05]\), set true null response rate \(p_0\) a
  pre-specified value.
\item
  \textbf{Step 2:} Let
  \[S_U =\min \{x \in \mathbb{N}:  PostP > \theta_U \}\] and
  \[ S_L =\max \{x \in \mathbb{N}:  PostP < \theta_L \}\] be the upper
  and lower decision boudries based on the number of observed responses.
\item
  \textbf{Step 3:} Make decisions after observing another \(x\)
  responses out of \(n\) patients:

  \begin{itemize}
  \tightlist
  \item
    If \(x \ge S_U\), then stop the trial for efficacy; (could be
    ignored for futility only)
  \item
    if \(x \le S_L\), then stop the trial for futility;
  \item
    otherwise, continue the trial until \(N_{max}\) reached.
  \end{itemize}
\end{itemize}

Although the stopping rule requires that the trial be terminated to
declare the experimental drug promising if \(x \ge S_U\), investigators
rarely stop the trial in such a case so that more patients are allowed
to benefit from the ``good'' drug. Therefore, the stopping rule for
superiority of the drug is often not implemented in a single-arm phase
II trial.

\subsubsection{\texorpdfstring{the cancer treatment neo-adjuvant therapy
and surgery with single pCR endpoint using \texttt{PostP}
method}{the cancer treatment neo-adjuvant therapy and surgery with single pCR endpoint using PostP method}}\label{the-cancer-treatment-neo-adjuvant-therapy-and-surgery-with-single-pcr-endpoint-using-postp-method}

First of all, we apply Simon's two-stage design:

\begin{Shaded}
\begin{Highlighting}[]
\KeywordTok{library}\NormalTok{(clinfun)}
\KeywordTok{ph2simon}\NormalTok{(}\DataTypeTok{pu=}\FloatTok{0.15}\NormalTok{, }\DataTypeTok{pa=}\FloatTok{0.30}\NormalTok{, }\DataTypeTok{ep1=}\FloatTok{0.05}\NormalTok{, }\DataTypeTok{ep2=}\FloatTok{0.10}\NormalTok{, }\DataTypeTok{nmax=}\DecValTok{100}\NormalTok{)}
\end{Highlighting}
\end{Shaded}

\begin{verbatim}
## 
##  Simon 2-stage Phase II design 
## 
## Unacceptable response rate:  0.15 
## Desirable response rate:  0.3 
## Error rates: alpha =  0.05 ; beta =  0.1 
## 
##         r1 n1  r  n EN(p0) PET(p0)
## Optimal  5 30 17 82  45.05  0.7106
## Minimax  6 42 14 64  51.80  0.5545
\end{verbatim}

Then we apply the PostP bayesian design by using the vague prior
\(Beta(1,1)\). (In the future, we can use informative priors)

\begin{Shaded}
\begin{Highlighting}[]
\KeywordTok{source}\NormalTok{(}\StringTok{"postp.R"}\NormalTok{)}
\KeywordTok{PostP.design}\NormalTok{(}\DataTypeTok{type =} \StringTok{"futility"}\NormalTok{, }\DataTypeTok{nmax=}\DecValTok{100}\NormalTok{, }\DataTypeTok{a=}\DecValTok{1}\NormalTok{, }\DataTypeTok{b=}\DecValTok{1}\NormalTok{, }\DataTypeTok{p0=}\FloatTok{0.15}\NormalTok{, }\DataTypeTok{delta=}\FloatTok{0.15}\NormalTok{, }\DataTypeTok{theta=}\FloatTok{0.05}\NormalTok{)}
\end{Highlighting}
\end{Shaded}

\begin{verbatim}
##     n bound
## 1   1    NA
## 8   8     0
## 13 13     1
## 18 18     2
## 23 23     3
## 27 27     4
## 32 32     5
## 36 36     6
## 40 40     7
## 44 44     8
## 48 48     9
## 52 52    10
## 56 56    11
## 60 60    12
## 64 64    13
## 68 68    14
## 72 72    15
## 76 76    16
## 80 80    17
## 84 84    18
## 88 88    19
## 92 92    20
## 95 95    21
## 99 99    22
\end{verbatim}

\begin{Shaded}
\begin{Highlighting}[]
\KeywordTok{PostP.design}\NormalTok{(}\DataTypeTok{type =} \StringTok{"efficacy"}\NormalTok{, }\DataTypeTok{nmax=}\DecValTok{100}\NormalTok{, }\DataTypeTok{a=}\DecValTok{1}\NormalTok{, }\DataTypeTok{b=}\DecValTok{1}\NormalTok{, }\DataTypeTok{p0=}\FloatTok{0.15}\NormalTok{, }\DataTypeTok{delta=}\FloatTok{0.15}\NormalTok{, }\DataTypeTok{theta=}\FloatTok{0.9}\NormalTok{)}
\end{Highlighting}
\end{Shaded}

\begin{verbatim}
##     n bound
## 1   1     1
## 3   3     2
## 7   7     3
## 12 12     4
## 17 17     5
## 22 22     6
## 27 27     7
## 32 32     8
## 37 37     9
## 42 42    10
## 48 48    11
## 53 53    12
## 59 59    13
## 64 64    14
## 70 70    15
## 76 76    16
## 81 81    17
## 87 87    18
## 93 93    19
## 99 99    20
\end{verbatim}

Theoretically, We can choose any one row as the early stopping. Here we
can compare the Simon's Optimal design result, and select r1/n1=5/32 as
the stopping rule for futility, r/n=17/81 as final stop for efficacy.
(or r1/n1/r/n = 6/36/14/64 sompared with Simon's Minimax).

We can also use Jeffrey prior: \(Beta(0.5,0.5)\).

\begin{Shaded}
\begin{Highlighting}[]
\KeywordTok{PostP.design}\NormalTok{(}\DataTypeTok{type =} \StringTok{"futility"}\NormalTok{, }\DataTypeTok{nmax=}\DecValTok{100}\NormalTok{, }\DataTypeTok{a=}\FloatTok{0.5}\NormalTok{, }\DataTypeTok{b=}\FloatTok{0.5}\NormalTok{, }\DataTypeTok{p0=}\FloatTok{0.15}\NormalTok{, }\DataTypeTok{delta=}\FloatTok{0.15}\NormalTok{, }\DataTypeTok{theta=}\FloatTok{0.05}\NormalTok{)}
\end{Highlighting}
\end{Shaded}

\begin{verbatim}
##     n bound
## 1   1    NA
## 6   6     0
## 12 12     1
## 17 17     2
## 22 22     3
## 26 26     4
## 30 30     5
## 35 35     6
## 39 39     7
## 43 43     8
## 47 47     9
## 51 51    10
## 55 55    11
## 59 59    12
## 63 63    13
## 67 67    14
## 71 71    15
## 75 75    16
## 79 79    17
## 83 83    18
## 87 87    19
## 91 91    20
## 94 94    21
## 98 98    22
\end{verbatim}

\begin{Shaded}
\begin{Highlighting}[]
\KeywordTok{PostP.design}\NormalTok{(}\DataTypeTok{type =} \StringTok{"efficacy"}\NormalTok{, }\DataTypeTok{nmax=}\DecValTok{100}\NormalTok{, }\DataTypeTok{a=}\FloatTok{0.5}\NormalTok{, }\DataTypeTok{b=}\FloatTok{0.5}\NormalTok{, }\DataTypeTok{p0=}\FloatTok{0.15}\NormalTok{, }\DataTypeTok{delta=}\FloatTok{0.15}\NormalTok{, }\DataTypeTok{theta=}\FloatTok{0.9}\NormalTok{)}
\end{Highlighting}
\end{Shaded}

\begin{verbatim}
##     n bound
## 1   1     1
## 3   3     2
## 6   6     3
## 11 11     4
## 15 15     5
## 20 20     6
## 25 25     7
## 30 30     8
## 35 35     9
## 41 41    10
## 46 46    11
## 52 52    12
## 57 57    13
## 63 63    14
## 68 68    15
## 74 74    16
## 80 80    17
## 85 85    18
## 91 91    19
## 97 97    20
\end{verbatim}

\subsection{\texorpdfstring{Predictive Probability Design
(\texttt{PredP})}{Predictive Probability Design (PredP)}}\label{predictive-probability-design-predp}

The predictive probability approach looks into the future objectives
based on the current observed data to project whether a positive
conclusion at the end of study is likely or not, and then makes a
sensible decision at the present time accordingly.(Lee and Liu 2008)

Let \(Y\) be the number of responses in the rest of \(m = N_{max} -n\)
future patients. Suppose our design is to declare efficacy if the
posterior probability of \(p\) exceeding some pre-specified level
\(p_0\) is greater than some threshold \(\theta_T\) . Marginalizing
\(p\) out of the binomial likelihood, it is well known that
Y\textbar{}X=x follows a beta-binomial distribution, i.e.
\(Y|x \sim Beta-Binomial(m, a + x, b + n - x)\).

By the end of the trial, suppose we observe additional \(Y = y\)
response, then the posterior distribution including future future y
patient \(p|(X = x, Y = y)\) is also \[Beta(a+x+y; b+N_{max}-x-y)\]. The
predictive probability (PredP) of trial success can then be calculated
as follows. Denote the posterior probability with the future data by
\(B_y = Pr(p > p0 | x, Y = y)\) and \(I_y = I(B_y > \theta_T )\), then
we have

\[ \begin{aligned} PredP & = Pr_{Y|x} \{Pr(p>p_0|x, Y) \ge \theta_T \} \\
        & = E \{ I[Pr(p>p_0|x, Y) \ge \theta_T] \Big| x  \}\\
        & = \sum\limits_{y=0}^{m}  I[Pr(p>p_0|x, Y=y) \ge \theta_T] \times Pr(Y=y|x)\\
        & = \sum\limits_{y=0}^{N_{max} -n}  I_y \times Pr(Y=y|x). \label{predp}
        \end{aligned} \]

Note that if there were no indicator function in \ref{predp}, the PredP
simply reduces to the PostP after averaging out the unobserved Y.

\[\sum\limits_{y=0}^{N_{max} -n}  Pr(p>p_0|x, Y=y) \times Pr(Y=y|x) = Pr(p>p_0|x)\]

Then we can decide the sample size and stop boundary by using the
following algorithm:

\subsubsection{Algorithm 2}\label{algorithm-2}

\begin{itemize}
\tightlist
\item
  \textbf{Step 1:} Specified the upper and lower probability cutoffs
  \(\theta_U\) and \(\theta_L\), typically, \(\theta_U \in [0.9,1]\) and
  \(\theta_L \in [0,0.05]\). Specified cutoff \(\theta_T\) for the
  future \(y\) patients, typically, \(\theta_T \in [0.8,1]\). Set true
  null response rate \(p_0\) a pre-specified value.
\item
  \textbf{Step 2:} Given \(x\) obwervations, let
  \[S_U =\min \{x+y \in \mathbb{N}:  PredP > \theta_U \}\] and
  \[ S_L =\max \{x+y \in \mathbb{N}:  PredP < \theta_L \}\] be the upper
  and lower decision boudries based on the number of observed responses.
\item
  \textbf{Step 3:} Make decisions after observing another \(x\)
  responses out of \(n\) patients:

  \begin{itemize}
  \tightlist
  \item
    If \(x \ge S_U\), then stop the trial for efficacy (could be ignored
    for futility only);
  \item
    if \(x \le S_L\), then stop the trial for futility;
  \item
    otherwise, continue the trial until \(N_{max}\) reached.
  \end{itemize}
\end{itemize}

\subsubsection{\texorpdfstring{the cancer treatment neo-adjuvant therapy
and surgery with single pCR endpoint using \texttt{PostP}
method}{the cancer treatment neo-adjuvant therapy and surgery with single pCR endpoint using PostP method}}\label{the-cancer-treatment-neo-adjuvant-therapy-and-surgery-with-single-pcr-endpoint-using-postp-method-1}

Now we can apply the PostP bayesian design to the the cancer treatment
pCR case, still use the vague prior \(Beta(1,1)\).

\begin{Shaded}
\begin{Highlighting}[]
\KeywordTok{source}\NormalTok{(}\StringTok{"predp.R"}\NormalTok{)}
\KeywordTok{PredP.design}\NormalTok{(}\DataTypeTok{type =} \StringTok{"futility"}\NormalTok{, }\DataTypeTok{nmax=}\DecValTok{100}\NormalTok{, }\DataTypeTok{a=}\DecValTok{1}\NormalTok{, }\DataTypeTok{b=}\DecValTok{1}\NormalTok{, }\DataTypeTok{p0=}\FloatTok{0.15}\NormalTok{, }\DataTypeTok{delta=}\FloatTok{0.15}\NormalTok{, }\DataTypeTok{theta=}\FloatTok{0.05}\NormalTok{)}
\end{Highlighting}
\end{Shaded}

\begin{verbatim}
##       n bound
## 1     1    NA
## 6     6     0
## 10   10     1
## 14   14     2
## 18   18     3
## 21   21     4
## 24   24     5
## 28   28     6
## 31   31     7
## 34   34     8
## 37   37     9
## 40   40    10
## 43   43    11
## 46   46    12
## 48   48    13
## 51   51    14
## 54   54    15
## 57   57    16
## 60   60    17
## 62   62    18
## 65   65    19
## 67   67    20
## 70   70    21
## 73   73    22
## 75   75    23
## 78   78    24
## 80   80    25
## 82   82    26
## 85   85    27
## 87   87    28
## 89   89    29
## 92   92    30
## 94   94    31
## 96   96    32
## 97   97    33
## 99   99    34
## 100 100    35
\end{verbatim}

\begin{Shaded}
\begin{Highlighting}[]
\KeywordTok{PredP.design}\NormalTok{(}\DataTypeTok{type =} \StringTok{"efficacy"}\NormalTok{, }\DataTypeTok{nmax=}\DecValTok{100}\NormalTok{, }\DataTypeTok{a=}\DecValTok{1}\NormalTok{, }\DataTypeTok{b=}\DecValTok{1}\NormalTok{, }\DataTypeTok{p0=}\FloatTok{0.15}\NormalTok{, }\DataTypeTok{delta=}\FloatTok{0.15}\NormalTok{, }\DataTypeTok{theta=}\FloatTok{0.9}\NormalTok{)}
\end{Highlighting}
\end{Shaded}

\begin{verbatim}
##     n bound
## 1   1     1
## 3   3     2
## 6   6     3
## 9   9     4
## 13 13     5
## 17 17     6
## 21 21     7
## 26 26     8
## 30 30     9
## 35 35    10
## 40 40    11
## 45 45    12
## 50 50    13
## 55 55    14
## 60 60    15
## 66 66    16
## 71 71    17
## 77 77    18
## 83 83    19
## 91 91    20
\end{verbatim}

Based on the Simon's Optimal design result, we can select r1/n1=5/24 as
the stopping rule for futility, r/n=17/71 as final stop for efficacy.
(or r1/n1/r/n = 6/28/14/55 sompared with Simon's Minimax). Compared with
Simon's design and PostP design, under the same stopping boudaries,
PredP design allows smaller sample sizes.

Using Jeffrey prior \(Beta(0.5,0.5)\) the results shown as follws:

\begin{Shaded}
\begin{Highlighting}[]
\KeywordTok{PredP.design}\NormalTok{(}\DataTypeTok{type =} \StringTok{"futility"}\NormalTok{, }\DataTypeTok{nmax=}\DecValTok{100}\NormalTok{, }\DataTypeTok{a=}\FloatTok{0.5}\NormalTok{, }\DataTypeTok{b=}\FloatTok{0.5}\NormalTok{, }\DataTypeTok{p0=}\FloatTok{0.15}\NormalTok{, }\DataTypeTok{delta=}\FloatTok{0.15}\NormalTok{, }\DataTypeTok{theta=}\FloatTok{0.05}\NormalTok{)}
\end{Highlighting}
\end{Shaded}

\begin{verbatim}
##       n bound
## 1     1    NA
## 4     4     0
## 9     9     1
## 13   13     2
## 17   17     3
## 20   20     4
## 24   24     5
## 27   27     6
## 30   30     7
## 33   33     8
## 36   36     9
## 39   39    10
## 42   42    11
## 45   45    12
## 48   48    13
## 51   51    14
## 54   54    15
## 57   57    16
## 59   59    17
## 62   62    18
## 65   65    19
## 67   67    20
## 70   70    21
## 72   72    22
## 75   75    23
## 78   78    24
## 80   80    25
## 82   82    26
## 85   85    27
## 87   87    28
## 89   89    29
## 91   91    30
## 94   94    31
## 96   96    32
## 97   97    33
## 99   99    34
## 100 100    35
\end{verbatim}

\begin{Shaded}
\begin{Highlighting}[]
\KeywordTok{PredP.design}\NormalTok{(}\DataTypeTok{type =} \StringTok{"efficacy"}\NormalTok{, }\DataTypeTok{nmax=}\DecValTok{100}\NormalTok{, }\DataTypeTok{a=}\FloatTok{0.5}\NormalTok{, }\DataTypeTok{b=}\FloatTok{0.5}\NormalTok{, }\DataTypeTok{p0=}\FloatTok{0.15}\NormalTok{, }\DataTypeTok{delta=}\FloatTok{0.15}\NormalTok{, }\DataTypeTok{theta=}\FloatTok{0.9}\NormalTok{)}
\end{Highlighting}
\end{Shaded}

\begin{verbatim}
##     n bound
## 1   1     1
## 2   2     2
## 5   5     3
## 9   9     4
## 12 12     5
## 16 16     6
## 21 21     7
## 25 25     8
## 30 30     9
## 34 34    10
## 39 39    11
## 44 44    12
## 49 49    13
## 54 54    14
## 60 60    15
## 65 65    16
## 71 71    17
## 77 77    18
## 83 83    19
## 90 90    20
\end{verbatim}

\subsubsection{MM \& APL data examples for PostP and PredP
design}\label{mm-apl-data-examples-for-postp-and-predp-design}

We also used the MM \& APL data to illustrate two bayesian methods by
animations.

Suppose we want to monitor the patients one-by-one, that is, the patient
outcome follows \(Bernoulli(p)\), and the stopping boundary can be
deciede based on updating posteriors. We can create the animation plots
to search the stopping boundary by taking the MM and APL data examples.

\begin{Shaded}
\begin{Highlighting}[]
\NormalTok{## Test MM data from the above examples}
\KeywordTok{bayes.desgin}\NormalTok{(}\DataTypeTok{mu0 =} \NormalTok{MM.mean, }\DataTypeTok{sigma =} \KeywordTok{sqrt}\NormalTok{(MM.var), }\DataTypeTok{r =} \NormalTok{MM.r, }\DataTypeTok{stop.rule =} \StringTok{"futility"}\NormalTok{, }
    \DataTypeTok{p0 =} \FloatTok{0.1}\NormalTok{, }\DataTypeTok{ymax =} \DecValTok{18}\NormalTok{)}
\end{Highlighting}
\end{Shaded}

\begin{verbatim}
## $para.a
## [1] 0.3
## 
## $para.b
## [1] 2.7
## 
## $`poterior mean`
##  [1] 0.0750 0.0600 0.0500 0.0429 0.0375 0.0333 0.0300 0.0273 0.0250 0.0231
## [11] 0.0214 0.0200
\end{verbatim}

\begin{verbatim}
## Loading required package: stats4
\end{verbatim}

\begin{verbatim}
## Loading required package: splines
\end{verbatim}

\begin{verbatim}
## [1] "Stop the trial for futility after the inclusion of 7 patients."
\end{verbatim}

\animategraphics[width=7in,controls,loop]{1}{vignette-BetaBinomial-pdf_files/figure-latex/unnamed-chunk-13-}{1}{13}

\begin{Shaded}
\begin{Highlighting}[]
\NormalTok{## Test APL data from the above examples}
\KeywordTok{bayes.desgin}\NormalTok{(}\DataTypeTok{mu0 =} \NormalTok{APL.mean, }\DataTypeTok{sigma =} \KeywordTok{sqrt}\NormalTok{(APL.var), }\DataTypeTok{r =} \NormalTok{APL.r, }\DataTypeTok{stop.rule =} \StringTok{"efficacy"}\NormalTok{, }
    \DataTypeTok{p0 =} \FloatTok{0.1}\NormalTok{, }\DataTypeTok{ymax =} \FloatTok{4.5}\NormalTok{)}
\end{Highlighting}
\end{Shaded}

\begin{verbatim}
## $para.a
## [1] 3
## 
## $para.b
## [1] 7
## 
## $`poterior mean`
##  [1] 0.273 0.333 0.308 0.286 0.333 0.375 0.412 0.444 0.421 0.450 0.476
## [12] 0.500 0.478 0.500 0.520 0.539 0.556 0.571 0.586 0.600
## 
## [1] "Stop the trial for efficacy after the inclusion of 10 patients."
\end{verbatim}

\animategraphics[width=7in,controls,loop]{1}{vignette-BetaBinomial-pdf_files/figure-latex/unnamed-chunk-14-}{1}{21}

\section{Summary and Future Works}\label{summary-and-future-works}

The frameworks of both \emph{PostP} and \emph{PredP} methods allow the
researcher to monitor the trial continously or by any cohort size.
Compared to Simon's minimax and optimal two-stage design, the PostP and
PredP designs monitor the data more frequently, both two designs have a
larger probability of early termination and a smaller expected sample
size in the null case than Simon's. All designs have the same maximum
sample size with controlled Type I and Type II error rates.

We can consider \(pCR\) not only binary level, but also multinomial
level, then we can use \textbf{Direchlet-Multinomial} prior and similar
idea to develop the single-arm design (Thall, Simon, and Estey 1995).
Another potential work is to extend the single primary endpoint to
co-primary endpoint by using similar Bayesian idea (or hierarchical
Bayesian design if the endpoint have some hierarchical structures)

\section*{References}\label{references}
\addcontentsline{toc}{section}{References}

\hypertarget{refs}{}
\hypertarget{ref-lee2008predictive}{}
Lee, J Jack, and Diane D Liu. 2008. ``A Predictive Probability Design
for Phase II Cancer Clinical Trials.'' \emph{Clinical Trials} 5 (2).
SAGE Publications: 93--106.

\hypertarget{ref-thall1994practical}{}
Thall, Peter F, and Richard Simon. 1994. ``Practical Bayesian Guidelines
for Phase IIB Clinical Trials.'' \emph{Biometrics}. JSTOR, 337--49.

\hypertarget{ref-thall1995bayesian}{}
Thall, Peter F, Richard M Simon, and Elihu H Estey. 1995. ``Bayesian
Sequential Monitoring Designs for Single-Arm Clinical Trials with
Multiple Outcomes.'' \emph{Statistics in Medicine} 14 (4). Wiley Online
Library: 357--79.

\hypertarget{ref-zohar2008bayesian}{}
Zohar, Sarah, Satoshi Teramukai, and Yinghui Zhou. 2008. ``Bayesian
Design and Conduct of Phase II Single-Arm Clinical Trials with Binary
Outcomes: A Tutorial.'' \emph{Contemporary Clinical Trials} 29 (4).
Elsevier: 608--16.

\end{document}
